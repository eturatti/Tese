% As configurações gerais são colocadas aqui, como novos comandos para o corpo do texto,
% informações de bookmark para o PDF, tamanho de parágrafos, entre outros.

% ----------------------------------------------------------
% Configurações do pacote BACKREF
% ----------------------------------------------------------
% Usado sem a opção hyperpageref de backref
\renewcommand{\backrefpagesname}{Citado na(s) página(s):~}
% Texto padrão antes do número das páginas
\renewcommand{\backref}{}
% Define os textos da citação
\renewcommand*{\backrefalt}[4]{
	\ifcase #1 %
		Nenhuma citação no texto.%
	\or
		Citado na página #2.%
	\else
		Citado #1 vezes nas páginas #2.%
	\fi}%
% ---

% ----------------------------------------------------------
% \theoremstyle{plain}


\newtheorem{mthm}{Main Theorem}
\numberwithin{mthm}{subsection}
\newtheorem{teorema}{Theorem}
\numberwithin{teorema}{subsection}
%\newtheorem{lemma}[theorem]{Lemma}
\newtheorem{proposition}{Proposition}
\numberwithin{proposition}{subsection}
\newtheorem{exer}{Exercise}
\numberwithin{exer}{subsection}
%\newtheorem{corollary}[theorem]{Corollary}
%\newtheorem{definition}[theorem]{Definition}
\newtheorem{theoremint}{Theorem}
\numberwithin{theoremint}{subsection}
%\newtheorem{definitionint}[theoremint]{Definition}
\newtheorem{conjecture}{Conjetura}
\numberwithin{conjecture}{subsection}
\newtheorem{cor}{Corollary}
\numberwithin{cor}{subsection}
\newtheorem{lema}{Lemma}
\numberwithin{lema}{subsection}

\theoremstyle{definition}
\newtheorem{definition}{Definition}
\numberwithin{definition}{subsection}

\newtheorem{Remark}{Observa\c{c}\~ao}
\numberwithin{Remark}{subsection}
%\newtheorem{Remarks}[theorem]{Remarks}
%\newtheorem{Examples}[theorem]{Examples}
%\newtheorem{Example}[theorem]{Example}

\newcommand{\cc}{\c{c}}
\newcommand{\Cc}{{\mathcal{C}}}
\newcommand{\opn}{{\mathcal{O}_{\mathbb{P}^n}}}
\newcommand{\pn}{{\mathbb{P}^n}}
\newcommand{\pnk}{{\mathbb{P}^n_K}}
\newcommand{\CC}{{\mathbb C}}
\newcommand{\KK}{{\mathbb K}}
\newcommand{\ZZ}{{\mathbb Z}}
\newcommand{\PP}{{\mathbb P}}
\newcommand{\GG}{{\mathbb G}}
\newcommand{\Aa}{{\mathbb A}}
\newcommand{\HH}{{\mathcal H}}
\newcommand{\UU}{{\mathcal U}}
\newcommand{\QQ}{{\mathbb Q}}
\newcommand{\MM}{{\mathcal M}}
\newcommand{\xdownarrow}[1]{%
	{\left\downarrow\vbox to #1{}\right.\kern-\nulldelimiterspace}
}
\newcommand{\A}{{\mathcal A}}
\newcommand{\BB}{{\mathcal B}}
\newcommand{\OO}{{\mathcal O}}
\newcommand{\Sym}{{\mathcal S}}
\newcommand{\gothm}{\mathfrak{m}}
\newcommand{\Div}{\operatorname{Div}}
\newcommand{\Pic}{\operatorname{Pic}}
\newcommand{\sing}{\operatorname{Sing}}
\newcommand{\car}{\operatorname{char}}
\newcommand{\rk}{\operatorname{rk}}
\newcommand{\Hom}{\operatorname{Hom}}
\newcommand{\End}{\operatorname{End}}
\newcommand{\tikzcircle}[2][red,fill=red]{\tikz[baseline=-0.5ex]\draw[#1,radius=#2] (0,0) circle ;}
\newcommand{\Ext}{\operatorname{Ext}}
\newcommand{\im}{\operatorname{Im}}
\newcommand{\codim}{\operatorname{codim}}
\newcommand{\coker}{\operatorname{coker}}
\newcommand{\Seg}{\operatorname{Seg}}
\DeclareMathOperator{\spec}{spec}
\DeclareMathOperator{\Spec}{Spec}
\DeclareMathOperator{\Proj}{Proj}
\DeclareMathOperator{\proj}{proj}
\makeatletter
\xpatchcmd{\@thm}{\thm@headpunct{.}}{\thm@headpunct{}}{}{}
\makeatother
\makeindex



\setlength{\textwidth}{15.5cm}
\setlength{\textheight}{21.5cm}
\setlength{\evensidemargin}{0cm}
\setlength{\oddsidemargin}{0cm}
\setlength{\topmargin}{0cm}

\usetikzlibrary{decorations.markings}
\tikzset{double line with arrow/.style args={#1,#2}{decorate,decoration={markings,%
			mark=at position 0 with {\coordinate (ta-base-1) at (0,1pt);
				\coordinate (ta-base-2) at (0,-1pt);},
			mark=at position 1 with {\draw[#1] (ta-base-1) -- (0,1pt);
				\draw[#2] (ta-base-2) -- (0,-1pt);
}}}}

% ----------------------------------------------------------

% ----------------------------------------------------------
% Conjunto de configuracoes para o pacote 'listings'
% ----------------------------------------------------------
\lstset{
  language=C++,
  basicstyle=\ttfamily, 
  keywordstyle=\color{blue}, 
  stringstyle=\color{verde}, 
  commentstyle=\color{red}, 
  extendedchars=true, 
  showspaces=false, 
  showstringspaces=false,
  numbers=left,
  numberstyle=\tiny,
  breaklines=true, 
  backgroundcolor=\color{green!10},
  breakautoindent=true,
  fontadjust=false
}
% ----------------------------------------------------------

% ----------------------------------------------------------
% Informações do PDF
% ----------------------------------------------------------
% Configurações de aparência do PDF final
% ---
% alterando o aspecto da cor azul
\definecolor{blue}{RGB}{41,5,195}
\definecolor{verde}{rgb}{0,0.5,0}
% ---
\makeatletter
\hypersetup{
  %pagebackref=true,
  pdftitle={\@title},
  pdfauthor={\@author},
  pdfsubject={%
    \imprimirtipotrabalho\ apresentada ao Instituto de Matemática, Estatística %
    e Computação Científica da Universidade Estadual de Campinas como parte dos %
    requisitos exigidos para a obtenção do título de \imprimirtitulacao\ em %
    \imprimircurso.
  },
  pdfcreator={LaTeX with unicamp-abnTeX2},
  pdfkeywords={abnt}{latex}{abntex}{abntex2}{trabalho acadêmico},
  colorlinks=true,		% false: boxed links; true: colored links
  linkcolor=blue,		% color of internal links
  citecolor=blue,		% color of links to bibliography
  filecolor=magenta,		% color of file links
  urlcolor=blue,		% color of internet links
  bookmarksdepth=4
}
\makeatother
% ----------------------------------------------------------

% ----------------------------------------------------------
% COMANDOS GLOBAIS
% ----------------------------------------------------------
\everymath{\displaystyle}
% ---
\renewcommand{\sin}{\mathrm{sen}}
\renewcommand{\tan}{\mathrm{tg}}
\renewcommand{\csc}{\mathrm{cossec}}
\renewcommand{\cot}{\mathrm{cotg}}
% ---
\DeclareMathOperator{\posto}{\mathrm{posto}}
\DeclareMathOperator{\conv}{\mathrm{conv}}
\DeclareMathOperator{\diag}{\mathrm{diag}}
\DeclareMathOperator{\argmin}{\mathrm{arg}\min}
\DeclareMathOperator{\argmax}{\mathrm{arg}\max}
% ---
% O tamanho do parágrafo é dado por:
\setlength{\parindent}{2.0cm}
% Controle do espaçamento entre um parágrafo e outro:
\setlength{\parskip}{0.2cm}  % tente também \onelineskip
% ---
% Para que apareça o nome 'Capítulo X' antes do título de cada capítulo
% \chapterstyle{default}
% ----------------------------------------------------------
\newsubfloat{figure}% Allow subfloats in figure environment
\providecommand*{\subfigureautorefname}{Subfigura}