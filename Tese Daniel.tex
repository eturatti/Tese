%% Customizações do abnTeX2 (http://www.abntex.net.br) para o Instituto de Matemática,
%% Estatística e Computação Científica da Universidade Estadual de Campinas (IMECC-UNICAMP)
%%
%% This work may be distributed and/or modified under the conditions of the LaTeX Project
%% Public License, either version 1.3 of this license or (at your option) any later version.
%% The latest version of this license is in http://www.latex-project.org/lppl.txt and version
%% 1.3 or later is part of all distributions of LaTeX version 2005/12/01 or later.
%%
%% This work has the LPPL maintenance status `maintained'.
%% 
%% The Current Maintainer of this work is Fábio Rodrigues Silva, gfabinhomat@gmail.com
%%
%% Further information about abnTeX2 are available on http://www.abntex.net.br
%%
\documentclass[
	oldfontcommands,
	% -- opções de customização --
	%sumario=tradicional,
	sumario=abnt-6027-2012,
	% -- opções da classe memoir --
	12pt,			% tamanho da fonte
	openright,		% capítulos começam em pág ímpar (insere página vazia caso preciso)
	oneside,		% para impressão em verso e anverso. Oposto a 'oneside'
	a4paper,		% tamanho do papel. 
	% -- opções da classe abntex2 --
	%chapter=TITLE,		% títulos de capítulos convertidos em letras maiúsculas
	%section=TITLE,		% títulos de seções convertidos em letras maiúsculas
	%subsection=TITLE,	% títulos de subseções convertidos em letras maiúsculas
	%subsubsection=TITLE,	% títulos de subsubseções convertidos em letras maiúsculas
	% -- opções do pacote babel --
	english,		% idioma adicional para hifenização
	brazil			% o último idioma é o principal do documento
	]{imecc-unicamp}
% ----------------------------------------------------------
% PACOTES ESSENCIAIS
% Aqui você pode adicionar seus pacotes específicos para uso em seu trabalho.
% Em PACOTES PESSOAIS insira os pacotes que desejar.
% ----------------------------------------------------------
% PACOTES BÁSICOS (ESSENCIAIS AO MODELO)
\usepackage{lmodern}			% Usa a fonte Latin Modern
\usepackage[T1]{fontenc}		% Selecao de codigos de fonte.
\usepackage[utf8]{inputenc}		% Codificacao do documento (conversão automática dos acentos)
\usepackage{lastpage}			% Usado pela Ficha catalográfica
\usepackage{indentfirst}		% Indenta o primeiro parágrafo de cada seção.
\usepackage{color,xcolor}		% Controle das cores
\usepackage{graphicx}			% Inclusão de gráficos
\usepackage{microtype} 			% para melhorias de justificação
% ----------------------------------------------------------
% ----------------------------------------------------------
% PACOTES PARA GLOSSÁRIO
\usepackage[subentrycounter,seeautonumberlist,nonumberlist=true]{glossaries}
% para usar o xindy ao invés do makeindex:
%\usepackage[xindy={language=portuguese},subentrycounter,seeautonumberlist,nonumberlist=true]{glossaries}
% ----------------------------------------------------------
% ----------------------------------------------------------
% PACOTES DE CITAÇÕES (PRINCIPAIS PACOTES DO MODELO)
\usepackage[brazilian]{backref}		% Paginas com as citações na bibl
\usepackage[alf,
	    abnt-repeated-author-omit=yes,
	    abnt-etal-list=0]{abntex2cite}		% Citações padrão ABNT
% É possível utilizar o sistema numérico de chamada, alterando a opção 'alf' para 'num'.
% Outros estilos bibliográficos podem ser usados. Se este for o caso, comente o pacote acima
% e utilize, por exemplo, o comando abaixo
% \bibliographystyle{acm}
% Consulte outros estilos de bibliografia consultando o manual de estilos bibliográficos do
% BibTeX em 'http://www.bibtex.org/'
% ----------------------------------------------------------
% ----------------------------------------------------------
% PACOTES ADICIONAIS (usados apenas no âmbito do Modelo Canônico do abnteX2)
\usepackage{lipsum}			% para geração de dummy text
% ----------------------------------------------------------
% ----------------------------------------------------------
% PACOTES PESSOAIS (USADOS PELO AUTOR -- acrescente aqui seus pacotes)\usepackage{bbm}
\usepackage{amsthm,amsfonts}
\usepackage{xpatch}
\usepackage[latin1]{inputenc}
\usepackage[T1]{fontenc}
\usepackage{amsthm}
\usepackage{makeidx}
\usepackage{xcolor,pict2e}
\usepackage{latexsym,amsmath,amsfonts,amscd,amssymb}
\usepackage{graphicx}
\usepackage{xypic}
\usepackage{array,booktabs,arydshln}
\usepackage{setspace}
\usepackage{fancyhdr}
\usepackage{fancybox}
\usepackage{multicol}
\usepackage[all]{xy}
\usepackage{pdfpages}
\usepackage{amsmath}
\usepackage{tikz-cd}
\usepackage{tikz}
\usetikzlibrary{cd}
% ----------------------------------------------------------
\usepackage[portuguese,onelanguage]{algorithm2e}	% para inserir algoritmos (longend,vlined)
% \usepackage{amsbsy}			% para símbolos matemáticos em negrito
% \usepackage{amscd}			% para diagramas
% \usepackage{amsfonts}			% fontes AMS
% \usepackage{amsmath}			% facilidades matemáticas
% \usepackage{amssymb}			% para os símbolos mais antigos
% \usepackage{amstext}			% para fragmentos tipo texto em modo matemático
\usepackage{amsthm}			% para teoremas
\usepackage{hyperref}			% Amplo suporte para hipertexto em LaTeX
\usepackage{cleveref}			% Referência cruzada inteligente
\usepackage{dsfont}			% para o estilo de conjuntos de números $\mathds{R}$
% \usepackage{ifthen}			% comandos de condição em LaTeX
\usepackage{listings}           	% para inserir códigos de outras linguagens de programação
% \usepackage{lscape}             	% para imprimir alguma página no formato paisagem
\usepackage{mathabx}			% conjunto de simbolos matemáticos
% \usepackage{mathrsfs}			% suporte para fontes RSFS
% \usepackage{pdfpages}           	% para inserir páginas PDF no texto

% \usepackage{verbatim}
% ----------------------------------------------------------
% ----------------------------------------------------------
% ----------------------------------------------------------
% INFORMAÇÕES E DADOS PARA CAPA E FOLHA DE ROSTO
% ----------------------------------------------------------
% Os comandos abaixo devem ser preenchidos em língua PORTUGUESA
\titulo{Estabilidade de Bridgeland e Problema de Interpola\cc\~ao}
\tipotrabalho{Dissertação}
% \tipotrabalho{Tese}
% \curso{Estatística}
% \curso{Matemática}
% \curso{Matemática Aplicada}
\curso{Matemática}
% \curso{}  % Se for aluno do PROFMAT
% ----------------------------------------------------------
% ----------------------------------------------------------
% Os comandos abaixo devem ser preenchidos na língua DO TRABALHO
% ---
% Se autor do sexo MASCULINO
 \autor{Ettore Teixeira Turatti}
% \titulacao{Mestre}
% \titulacao{Doutor}
% ---
% Se autor do sexo FEMININO, descomente e preencha os campos abaixo
%\autor[autora]{Nome Completo da Aluna}
%\titulacao{Mestra}
% \titulacao{Doutora}
% ---
% Se orientador/coorientador do sexo MASCULINO
 \orientador{Simone Marchesi}
% \coorientador{Nome Completo do Coorientador}
% ---
% Se orientador/coorientador do sexo FEMININO, descomente e preencha os campos abaixo (exceto se o trabalho for em inglês)
%\orientador[Orientadora]{Nome Completo da Orientadora}
%\coorientador[Coorientadora]{Nome Completo da Coorientadora}
% ---
\data{2019}
% ---
% Se seu trabalho for em língua NÃO portuguesa, descomente e preencha os campos abaixo na lingua DO TRABALHO
 \setboolean{ABNTEXotherlanguage}{true}
 \titulootherlang{Title of your Academic Work:\\ master thesis or phd thesis}
% \cursootherlang{Statistics}
 \cursootherlang{Mathematics}
% \cursootherlang{Applied Mathematics}
% \cursootherlang{Computational and Applied Mathematics}
 \typework{Dissertation}
% \typework{Thesis}
 \titulation{Master}
% \titulation{Doctor}
% ---
% No caso de Cotutela Internacional de Tese, descomente e preencha os campos abaixo
% \setboolean{ABNTEXcotutela}{true}
% \universidadecotutela{NAME OF THE UNIVERSITY}
% \paiscotutela{COUNTRY}
% ----------------------------------------------------------
% ----------------------------------------------------------
% ----------------------------------------------------------
% CONFIGURAÇÕES GERAIS
% As configurações gerais são colocadas aqui, como novos comandos para o corpo do texto,
% informações de bookmark para o PDF, tamanho de parágrafos, entre outros.

% ----------------------------------------------------------
% Configurações do pacote BACKREF
% ----------------------------------------------------------
% Usado sem a opção hyperpageref de backref
\renewcommand{\backrefpagesname}{Citado na(s) página(s):~}
% Texto padrão antes do número das páginas
\renewcommand{\backref}{}
% Define os textos da citação
\renewcommand*{\backrefalt}[4]{
	\ifcase #1 %
		Nenhuma citação no texto.%
	\or
		Citado na página #2.%
	\else
		Citado #1 vezes nas páginas #2.%
	\fi}%
% ---

% ----------------------------------------------------------
% \theoremstyle{plain}


\newtheorem{mthm}{Main Theorem}
\numberwithin{mthm}{subsection}
\newtheorem{teorema}{Theorem}
\numberwithin{teorema}{subsection}
%\newtheorem{lemma}[theorem]{Lemma}
\newtheorem{proposition}{Proposition}
\numberwithin{proposition}{subsection}
\newtheorem{exer}{Exercise}
\numberwithin{exer}{subsection}
%\newtheorem{corollary}[theorem]{Corollary}
%\newtheorem{definition}[theorem]{Definition}
\newtheorem{theoremint}{Theorem}
\numberwithin{theoremint}{subsection}
%\newtheorem{definitionint}[theoremint]{Definition}
\newtheorem{conjecture}{Conjetura}
\numberwithin{conjecture}{subsection}
\newtheorem{cor}{Corollary}
\numberwithin{cor}{subsection}
\newtheorem{lema}{Lemma}
\numberwithin{lema}{subsection}

\theoremstyle{definition}
\newtheorem{definition}{Definition}
\numberwithin{definition}{subsection}

\newtheorem{Remark}{Observa\c{c}\~ao}
\numberwithin{Remark}{subsection}
%\newtheorem{Remarks}[theorem]{Remarks}
%\newtheorem{Examples}[theorem]{Examples}
%\newtheorem{Example}[theorem]{Example}

\newcommand{\cc}{\c{c}}
\newcommand{\Cc}{{\mathcal{C}}}
\newcommand{\opn}{{\mathcal{O}_{\mathbb{P}^n}}}
\newcommand{\pn}{{\mathbb{P}^n}}
\newcommand{\pnk}{{\mathbb{P}^n_K}}
\newcommand{\CC}{{\mathbb C}}
\newcommand{\KK}{{\mathbb K}}
\newcommand{\ZZ}{{\mathbb Z}}
\newcommand{\PP}{{\mathbb P}}
\newcommand{\GG}{{\mathbb G}}
\newcommand{\Aa}{{\mathbb A}}
\newcommand{\HH}{{\mathcal H}}
\newcommand{\UU}{{\mathcal U}}
\newcommand{\QQ}{{\mathbb Q}}
\newcommand{\MM}{{\mathcal M}}
\newcommand{\xdownarrow}[1]{%
	{\left\downarrow\vbox to #1{}\right.\kern-\nulldelimiterspace}
}
\newcommand{\A}{{\mathcal A}}
\newcommand{\BB}{{\mathcal B}}
\newcommand{\OO}{{\mathcal O}}
\newcommand{\Sym}{{\mathcal S}}
\newcommand{\gothm}{\mathfrak{m}}
\newcommand{\Div}{\operatorname{Div}}
\newcommand{\Pic}{\operatorname{Pic}}
\newcommand{\sing}{\operatorname{Sing}}
\newcommand{\car}{\operatorname{char}}
\newcommand{\rk}{\operatorname{rk}}
\newcommand{\Hom}{\operatorname{Hom}}
\newcommand{\End}{\operatorname{End}}
\newcommand{\tikzcircle}[2][red,fill=red]{\tikz[baseline=-0.5ex]\draw[#1,radius=#2] (0,0) circle ;}
\newcommand{\Ext}{\operatorname{Ext}}
\newcommand{\im}{\operatorname{Im}}
\newcommand{\codim}{\operatorname{codim}}
\newcommand{\coker}{\operatorname{coker}}
\newcommand{\Seg}{\operatorname{Seg}}
\DeclareMathOperator{\spec}{spec}
\DeclareMathOperator{\Spec}{Spec}
\DeclareMathOperator{\Proj}{Proj}
\DeclareMathOperator{\proj}{proj}
\makeatletter
\xpatchcmd{\@thm}{\thm@headpunct{.}}{\thm@headpunct{}}{}{}
\makeatother
\makeindex



\setlength{\textwidth}{15.5cm}
\setlength{\textheight}{21.5cm}
\setlength{\evensidemargin}{0cm}
\setlength{\oddsidemargin}{0cm}
\setlength{\topmargin}{0cm}

\usetikzlibrary{decorations.markings}
\tikzset{double line with arrow/.style args={#1,#2}{decorate,decoration={markings,%
			mark=at position 0 with {\coordinate (ta-base-1) at (0,1pt);
				\coordinate (ta-base-2) at (0,-1pt);},
			mark=at position 1 with {\draw[#1] (ta-base-1) -- (0,1pt);
				\draw[#2] (ta-base-2) -- (0,-1pt);
}}}}

% ----------------------------------------------------------

% ----------------------------------------------------------
% Conjunto de configuracoes para o pacote 'listings'
% ----------------------------------------------------------
\lstset{
  language=C++,
  basicstyle=\ttfamily, 
  keywordstyle=\color{blue}, 
  stringstyle=\color{verde}, 
  commentstyle=\color{red}, 
  extendedchars=true, 
  showspaces=false, 
  showstringspaces=false,
  numbers=left,
  numberstyle=\tiny,
  breaklines=true, 
  backgroundcolor=\color{green!10},
  breakautoindent=true,
  fontadjust=false
}
% ----------------------------------------------------------

% ----------------------------------------------------------
% Informações do PDF
% ----------------------------------------------------------
% Configurações de aparência do PDF final
% ---
% alterando o aspecto da cor azul
\definecolor{blue}{RGB}{41,5,195}
\definecolor{verde}{rgb}{0,0.5,0}
% ---
\makeatletter
\hypersetup{
  %pagebackref=true,
  pdftitle={\@title},
  pdfauthor={\@author},
  pdfsubject={%
    \imprimirtipotrabalho\ apresentada ao Instituto de Matemática, Estatística %
    e Computação Científica da Universidade Estadual de Campinas como parte dos %
    requisitos exigidos para a obtenção do título de \imprimirtitulacao\ em %
    \imprimircurso.
  },
  pdfcreator={LaTeX with unicamp-abnTeX2},
  pdfkeywords={abnt}{latex}{abntex}{abntex2}{trabalho acadêmico},
  colorlinks=true,		% false: boxed links; true: colored links
  linkcolor=blue,		% color of internal links
  citecolor=blue,		% color of links to bibliography
  filecolor=magenta,		% color of file links
  urlcolor=blue,		% color of internet links
  bookmarksdepth=4
}
\makeatother
% ----------------------------------------------------------

% ----------------------------------------------------------
% COMANDOS GLOBAIS
% ----------------------------------------------------------
\everymath{\displaystyle}
% ---
\renewcommand{\sin}{\mathrm{sen}}
\renewcommand{\tan}{\mathrm{tg}}
\renewcommand{\csc}{\mathrm{cossec}}
\renewcommand{\cot}{\mathrm{cotg}}
% ---
\DeclareMathOperator{\posto}{\mathrm{posto}}
\DeclareMathOperator{\conv}{\mathrm{conv}}
\DeclareMathOperator{\diag}{\mathrm{diag}}
\DeclareMathOperator{\argmin}{\mathrm{arg}\min}
\DeclareMathOperator{\argmax}{\mathrm{arg}\max}
% ---
% O tamanho do parágrafo é dado por:
\setlength{\parindent}{2.0cm}
% Controle do espaçamento entre um parágrafo e outro:
\setlength{\parskip}{0.2cm}  % tente também \onelineskip
% ---
% Para que apareça o nome 'Capítulo X' antes do título de cada capítulo
% \chapterstyle{default}
% ----------------------------------------------------------
\newsubfloat{figure}% Allow subfloats in figure environment
\providecommand*{\subfigureautorefname}{Subfigura}
% ----------------------------------------------------------
% ----------------------------------------------------------
% COMPILA O ÍNDICE (OPCIONAL)
\makeindex
% ----------------------------------------------------------
% ----------------------------------------------------------
% CONTÉM TODAS AS ENTRADAS DO GLOSSÁRIO (OPCIONAL)
% ----------------------------------------------------------
% COMPILA O GLOSSÁRIO
\makeglossaries
% ----------------------------------------------------------
% ----------------------------------------------------------
% ENTRADAS DO GLOSSÁRIO
\newglossaryentry{pai}{
    name={pai},
    plural={pai},
    description={este é uma entrada pai, que possui outras
    subentradas.}
}
\newglossaryentry{componente}{
    name={componente},
    plural={componentes},
    parent=pai,
    description={descriação da entrada componente.}
}
\newglossaryentry{filho}{
    name={filho},
    plural={filhos},
    parent=pai,
    description={isto é uma entrada filha da entrada de nome
	\gls{pai}. Trata-se de uma entrada irmã da entrada
	\gls{componente}.
    }
}
\newglossaryentry{equilibrio}{
    name={equilíbrio da configuração},
    see=[veja também]{componente},
    description={consistência entre os \glspl{componente}}
}
\newglossaryentry{latex}{
    name={LaTeX},
    description={ferramenta de computador para autoria de
    documentos criada por D. E. Knuth}
}
\newglossaryentry{abntex2}{
    name={abnTeX2},
    see=latex,
    description={suíte para LaTeX que atende os requisitos das
    normas da ABNT para elaboração de documentos técnicos e
    científicos brasileiros}
}
% ----------------------------------------------------------
% ----------------------------------------------------------
% EXEMPLO DE CONFIGURAÇÃO DO GLOSSÁRIO
\renewcommand*{\glsseeformat}[3][\seename]{\textit{#1}  
 \glsseelist{#2}}
% ----------------------------------------------------------
% ----------------------------------------------------------
% Define nome e preâmbulo do glossário
\renewcommand{\glossaryname}{Glossário}
% \renewcommand{\glossarypreamble}{Esta é a descrição do glossário. Experimente
% visualizar outros estilos de glossários, como o \texttt{altlisthypergroup},
% por exemplo.\\
% \\}
% ----------------------------------------------------------
% ----------------------------------------------------------
% Traduções para o ambiente glossaries
\providetranslation{Glossary}{Glossário}
\providetranslation{Acronyms}{Siglas}
\providetranslation{Notation (glossaries)}{Notação}
\providetranslation{Description (glossaries)}{Descrição}
\providetranslation{Symbol (glossaries)}{Símbolo}
\providetranslation{Page List (glossaries)}{Lista de Páginas}
\providetranslation{Symbols (glossaries)}{Símbolos}
\providetranslation{Numbers (glossaries)}{Números} 
% ----------------------------------------------------------
% ----------------------------------------------------------
% Estilo de glossário
\setglossarystyle{index}
% \setglossarystyle{altlisthypergroup}
% \setglossarystyle{tree}
% ----------------------------------------------------------
% ----------------------------------------------------------
% -----------------------------------------------------------------------------------------------
% %%%%%%%%%%%%%%%%%%%%%%%%%%%%%%%%%%%%% INÍCIO DO DOCUMENTO %%%%%%%%%%%%%%%%%%%%%%%%%%%%%%%%%%%%%
% -----------------------------------------------------------------------------------------------
\begin{document}
% Seleciona o idioma do documento (conforme pacotes do babel)
\selectlanguage{english}
% \selectlanguage{english}
% Retira espaço extra obsoleto entre as frases.
\frenchspacing
% ---------------------------------------------------------------------------------
% %%%%%%%%%%%%%%%%%%%%%%% INÍCIO DOS ELEMENTOS PRÉ-TEXTUAIS %%%%%%%%%%%%%%%%%%%%%%%
% ---------------------------------------------------------------------------------
\pretextual
% ----------------------------------------------------------
% PRIMEIRA FOLHA (OBRIGATÓRIO)
\imprimirprimeirafolha
% ----------------------------------------------------------
% ----------------------------------------------------------
% FOLHA DE ROSTO (OBRIGATÓRIO)
% ---
% Após realizar as correções finais de seu trabalho acadêmico, escaneie a folha de rosto
% devidamente assinada pelo orientador e salve no formato PDF com o nome 'folha-de-rosto.pdf'
% no diretório do seu projeto. Daí substitua a linha do comando '\imprimirfolhaderosto'
% pelas 3 linhas de comando abaixo:
% ---
 \begin{folhaderosto}
     \includepdf{folha-de-rosto.pdf}
 \end{folhaderosto}
% ---
%\imprimirfolhaderosto
% ----------------------------------------------------------
% ----------------------------------------------------------
% FICHA CATALOGRÁFICA (OBRIGATÓRIO)
% ---
% A biblioteca da UNICAMP lhe fornecerá um PDF com a ficha catalográfica definitiva após a defesa
% do trabalho {http://hamal.bc.unicamp.br/catalogonline2/}. Quando estiver com o documento, salve-o
% como PDF no diretório do seu projeto e deixe apenas o comando '\includepdf{ficha-catalografica.pdf}'
% dentro do ambiente abaixo
% ---
%\begin{fichacatalografica}
%    \begin{center}
%	{\ABNTEXchapterfont\large A ficha catalográfica será fornecida pela biblioteca}
%    \end{center}
\includepdf{ficha-catalografica.pdf}
%\end{fichacatalografica}
% ----------------------------------------------------------
% ----------------------------------------------------------
% FOLHA DE APROVAÇÃO (OBRIGATÓRIO)
% ---
% A folha de aprovação será fornecida pela secretaria de pós-graduação. Após recebê-la, escaneie a folha
% salvando em PDF no diretório do seu projeto com o nome 'folhadeaprovacao.pdf' e deixe apenas o comando
% '\includepdf{folhadeaprovacao.pdf}' dentro do ambiente abaixo
% ---
%\begin{folhadeaprovacao}
%\centering{\ABNTEXchapterfont\large A folha de aprovação será %fornecida pela Secretaria de Pós-Graduação}
\includepdf{folhadeaprovacao.pdf}
%\end{folhadeaprovacao}
% ----------------------------------------------------------
% ----------------------------------------------------------
% DEDICATÓRIA (OPCIONAL)
\begin{dedicatoria}
   \vspace*{\fill}
   \centering
   \noindent
   \textit{
A todos aqueles que não puderam seguir a carreira na matemática
   }
   \vspace*{\fill}
\end{dedicatoria}
% ----------------------------------------------------------
% ----------------------------------------------------------
% AGRADECIMENTOS (OPCIONAL)
\begin{agradecimentos}
Gostaria de agradecer a todos que me apoiaram. Em especial meus pais, Marina e Itsuo Futata, que sempre fizeram o máximo possível por qualquer aspecto da minha vida. Minha irmã Beatriz Futata por me receber com uma quantidade absurda de doces e atenção. E minhas outras irmãs Giovanna de Luca e Regiane Matsumoto pelos anos de amizade e por me receber com uma quantidade absurda de suco e nuggets entre outras comidas.

Aos meus amigos Victor Pretti, Renato Silva e Lucas Figueiredo pelos momentos de insanidade, agostos passados juntos e vídeos de gatinhos. Espero muitos funko's pop pela frente. Ao Ricardo Douglas Costa também pelos eventos mais inusitados possíveis, eternamente uma lenda.

Aos amigos Marcus Santos, Gabriel Aranha e André Rodrigues por nunca esquecerem o que temos, mesmo nas condições de distância e até mesmo tempo em reclusão. Estou esperando ainda nossa escola.

Ao Guterman Junior por me ensinar mais do que eu gostaria sobre carros e também por me ensinar como me organizar, coisa que nunca aprendi. À Laryssa Abdala por me mostrar óticas completamente diferentes da vida, muitas delas eu sigo até hoje.

Ao Matheus Gomes, meu eterno vizinho e amigo. Pelos filmes românticos e companhia em um período difícil. Nenhum prato vai superar nosso pão de queijo, mas qualquer coisa supera nossa coxinha frita em azeite. 

Ao meu orientador Simone Marchesi, que me ensinou como andar no meio acadêmico e estava sempre presente para dar suporte. E ao professor Marcos Jardim pelas discussões inestimáveis que foram essenciais para o mestrado.

Um especial agradecimento à secretaria do IMECC pela paciência em me guiar nos processos burocráticos, vocês fazem muito pelos alunos!

Finalizo agradecendo ao CNPq por ter financiado meu projeto de mestrado.
\end{agradecimentos}
% ----------------------------------------------------------
% ----------------------------------------------------------
% EPÍGRAFE (OPCIONAL)
\begin{epigrafe}
    \vspace*{\fill}
    \begin{flushright}
	\textit{"To show or to be shown?" is a question never,
		even not known by many to exist - The Residents
	}
    \end{flushright}
\end{epigrafe}
% ----------------------------------------------------------
% ----------------------------------------------------------
% RESUMOS (OBRIGATÓRIO)

% -------------------------------------------------------------
%  RESUMOS
\setlength{\absparsep}{18pt} % ajusta o espaçamento dos parágrafos do resumo
% -------------------------------------------------------------
% ATENÇÃO: o ambiente 'otherlanguage*' deve ser usado para o resumo que não está na
% língua vernácula do trabalho, com a respectiva opção linguística do pacote 'babel'.
% -------------------------------------------------------------
% resumo em PORTUGUÊS (OBRIGATÓRIO)
\begin{resumo}[Resumo]
 \begin{otherlanguage*}{brazil}
    Segundo a [3.1-3.2]{NBR6028:2003}, o resumo deve ressaltar o objetivo,
    o método, os resultados e as conclusões do documento. A ordem e a extensão destes
    itens dependem do tipo de resumo (informativo ou indicativo) e do tratamento que
    cada item recebe no documento original. O resumo deve ser precedido da referência
    do documento, com exceção do resumo inserido no próprio documento. (\ldots) As
    palavras-chave devem figurar logo abaixo do resumo, antecedidas da expressão
    Palavras-chave:, separadas entre si por ponto e finalizadas também por ponto.

    \textbf{Palavras-chave}: latex. abntex. editoração de texto.
 \end{otherlanguage*}
\end{resumo}
% -------------------------------------------------------------
% -------------------------------------------------------------
% resumo em INGLÊS (OBRIGATÓRIO)
\begin{resumo}[Abstract]
 \begin{otherlanguage*}{english}
    This is the english abstract.
    
    \textbf{Keywords}: latex. abntex. text editoration.
 \end{otherlanguage*}
\end{resumo}
% -------------------------------------------------------------

% ----------------------------------------------------------
% ----------------------------------------------------------
% ----------------------------------------------------------
% ----------------------------------------------------------
% ----------------------------------------------------------
% ----------------------------------------------------------
% ----------------------------------------------------------
% ----------------------------------------------------------
% ----------------------------------------------------------
% ----------------------------------------------------------
% SUMÁRIO (OBRIGATÓRIO)
\pdfbookmark[0]{\contentsname}{toc}
\tableofcontents*
\cleardoublepage
% ----------------------------------------------------------
% ---------------------------------------------------------------------------------
% %%%%%%%%%%%%%%%%%%%%%%%% FIM DOS ELEMENTOS PRÉ-TEXTUAIS %%%%%%%%%%%%%%%%%%%%%%%%
% ---------------------------------------------------------------------------------
% ---------------------------------------------------------------------------------
% %%%%%%%%%%%%%%%%%%%%%%%%% INÍCIO DOS ELEMENTOS TEXTUAIS %%%%%%%%%%%%%%%%%%%%%%%%%
% ---------------------------------------------------------------------------------
\textual
% ----------------------------------------------------------
% INTRODUÇÃO
% ----------------------------------------------------------
% Exemplo de capítulo sem numeração, mas presente no Sumário
\chapter*[Introdução]{Introdução}
\addcontentsline{toc}{chapter}{Introdução}
% ----------------------------------------------------------

Polynomial interpolation is an interesting problem in mathematics, as it has several applications to pure and applied math, we can cite as examples the aproximation of complicated functions by a polynomials, like the Taylor's expansion $$
f(x)=\sum_{n=0}^{r}\frac{f^{(n)}(a)}{n!}(x-a)^n+E_n(x),
$$ or the numerical quadrature. The classical problem is how to find a polynomial passing through a given set of points and its solution in $\mathbb R^2$ is quite simple, given points $(x_1,y_1),\dots,(x_n,y_n)$ the polynomial $$
p(x)=\sum_{i=0}^n\bigg[\prod_{\substack{0\le j\le n\\j\neq i}}\frac{x-x_j}{x_i-x_j}\bigg]y_i,
$$ sattisfies the propety, i.e., $p(x_i)=y_i$. In algebraic geometry such problem can be described in terms of cohomology, given a set of points, namely $Z$, when is it possible to find a line bundle such that $H^i(I_Z)=0$ for all $i\geq 0$. This natural problem can be extended to higher-rank bundles, i.e., when one can find a vector bundle $E$ with $H^i(I_Z=0)$ for all $i\geq 0$. 

Stability conditions on triangulated categories were introduced in \cite{Bridgeland} by Tom Bridgeland, inspired by the work done in \cite{Douglas:2002fj} by Michael R. Douglas of $2002$ on string theory. The main result proved by Bridgland in this first paper is that on a fixed category $\mathcal D$ one can associate a complex manifold $Stab(\mathcal D)$ parametrizing the set of stability conditions on $\mathcal D$. One of the first reasons that motivated the study of the space of stability conditions was that they define a new invariant for triangulated categories. Also, it is shown in \cite{bridgeland2008} the profound connection between geometrical ideas and homological algebra. 

Calculating cohomologies is not an easy task in general, so Coskun and Huizenga shows in \cite{COSKUN} that there is a deep connection between the interpolation problem and Bridgeland stability. In the light of the work done in \cite{ARCARA2013580}, where it is shown that the moduli space of Bridgland semi-stable objects are isomorphic to moduli space of quiver representations, furthermore this shows the finiteness of Bridgeland walls. Not only that, but it is also studied what is the destabilizing object of a zero-dimensional $Z$, and the relation between Bridgeland walls and the walls in the stable base locus decomposition. 

These results leads to the Proposition \ref{prop 4}, that gives the correspondence between the geometry of a Bridgeland potential wall defined by two Chern characters $\xi_1,\xi_2$ and the numerical invariants of a vector bundle $\zeta$ orthogonal to the objects defining the wall, where it is shown that the center and the radius of such potential wall will be the respectively the slope and the discriminant of $\zeta$. Considering this, Coskun and Huizenga proposes that finding the destabilizing sequence of a zero-dimensional scheme, also means finding an exact sequence that has acyclic side terms when tensored by $\zeta$ (or the general element $E$ of the stack of prioritary sheaves with Chern character equals to $\zeta$), therefore solving $H^i(E\otimes I_Z)=0$.

Furthermore, \ref{bigger slope theorem} shows that if we find a general bundle $E$ with slope $\mu$ that satisfies interpolation for $Z$, then the general bundle of the stack of prioritary bundles with Chern character $\xi$, that has slope $\mu'\geq\mu$ also satisfies interpolation for $Z$, therefore the question is not only finding bundles that satisfies interpolation, but also finding the smaller slope such that a general bundle of some Chern character with that slope satisfies interpolation. 



Again, the destabilizing sequence gives the answer. In \cite{COSKUN} it is shown that for $Z$ a complete intersection scheme or a monomial scheme, the orthogonal Chern character to both the ideal sheaf of the scheme and the destabilizing object not only gives an acyclical sequences when tensorizing the destabilizing sequence, but also the general element of the stack of prioritary sheaves of such Chern character is the one having minimal slope satisfying interpolation for $Z$.

The dissertation is structured in two parts.

The first chapter is an introduction to the tools necessaries to follow the text. The first section introduces the notion of sheaves and schemes, the notion of divisors and Chern classes, we finish it with the \textit{Hizerbuch-Riemman-Roch} theorem. The second section is an introduction to derived categories. In the third section is presented the notion of cohomology for a sheave and the \textit{Serre duality} theorem. The last section is an introduction to stacks.

In the second chapter is presented the study of \cite{COSKUN}. On the first section it is defined the notion of a Bridgeland stability, we also define a Bridgeland stability suitable for the interpolation problem and calculate the behavior of the potential Bridgeland walls. We also prove Theorem \ref{bigger slope theorem} and give the candidate for the minimal slope that satisfies interpolation. The second section is dedicated to prove the interpolation problem for complete intersection schemes. The next section is an introduction to monomial schemes, where we introduce its representation as a block diagram, calculate invariants of the potential Bridgeland walls of such schemes and its destabilizing walls. The last section we prove interpolation problem for monomial schemes.

% ----------------------------------------------------------
% DESENVOLVIMENTO
% \part{Preparação da pesquisa}
% \include{capitulo1}
% \include{capitulo2}
% \part{Referenciais teóricos}
% \include{capitulo3}
% \include{capitulo4}
% \include{capitulo5}
% \part{Resultados}
% \include{capitulo6}
% \include{capitulo7}
% --------------------
% INÍCIO DOS EXEMPLOS
% --------------------
% No âmbito do Modelo Canônico, os comandos a seguir apresentam um sucinto resumo de como
% esta classe pode ser usada. Oriente a construção do seu trabalho com base nestes comandos.
% ---
%\include{conteudo} TIRAR % DEPOIS
\chapter{Preliminaries}
%\chapter{Definitions and Preliminaries}
In this chapter we will recall the necessary concepts and results about coherent sheaves, monads, curves and Chern classes, including the Hirzebruch-Riemann-Roch theorem, which we will need throughout this dissertation.
\section{Basic definitions and results}
$X$ denotes a scheme, from now on whenever we write $\mathcal{F}^I$ for a sheaf $\mathcal{F}$ we mean the direct sum $\bigoplus_{i \in I} \mathcal{F}_i$ where $\mathcal{F}_i=\mathcal{F}$. If $r \in \mathbb{N}$ we will write $\mathcal{F}^r$ to denote the direct sum of $r$ copies of $\mathcal{F}$.
The base field $k$ is an algebraically closed field of characteristic 0.
\begin{definition}
A sheaf of $\mathcal{O}_X$-modules $\mathcal{F}$ is \\
(i) finitely generated, or of finite type if every point $x \in X$ has an open neighbourhood such that there is a surjective morphism
\begin{equation}
\mathcal{O}_X^n|_U \to \mathcal{F}|_U
\end{equation}
with $n$ finite \\
(ii) coherent if it is finitely generated and for every open $U$, every finite $p \in \mathbb{N}$ and every morphism 
\begin{equation}
\mathcal{O}_X^p|_U \to \mathcal{F}|_U
\end{equation}
of $\mathcal{O}_X|_U$-modules has a finitely generated kernel. \\
(iii) finitely presented if there is an exact sequence of the form
\begin{equation}
\mathcal{O}_X^p \to \mathcal{O}_X^n \to \mathcal{F} \to 0
\end{equation}
with $p$ and $n$ finite. Every finitely presented $\mathcal{O}_X$-module is finitely generated. \\
(iv) quasi coherent if it is locally presentable, i.e. there is an open cover $\{U_i \}$ of $X$ and for every $i$ an exact sequence
\begin{equation}
\mathcal{O}_X^{I_i}|_{U_i} \to \mathcal{O}_X^{J_i}|_{U_i} \to \mathcal{F}|_{U_i} \to 0
\end{equation}
where $I_i$ and $J_i$ may be infinite.
\end{definition}

If $X$ is locally noetherian, those concepts become intertwined :
\begin{lemma}\cite[Stacks, 29.9.1]{stacks-project}\label{locally noetherian}
Let $X$ be a locally noetherian scheme. Let $\mathcal{F}$ be an $\mathcal{O}_X$-module. The following are equivalent \\
(i) $\mathcal{F}$ is coherent, \\
(ii) $\mathcal{F}$ is a quasi-coherent, finite-type $\mathcal{O}_X$-module, \\
(iii) $\mathcal{F}$ is a finitely presented $\mathcal{O}_X$-module, \\
(iv) for any affine open $Spec(A)=U \subset X$ we have $\mathcal{F}|_U = \bar{M}$ with $M$ a finite $A-module$, \\
(v) there exists an affine open covering $X= \bigcup U_i$, $U_i = Spec(A_i)$ such that each $\mathcal{F}|_{U_i}=\bar{M_i}$ with $M_i$ a finite $A_i$-module.
\end{lemma}
We will always deal with locally noetherian ringed spaces. So (\ref{locally noetherian}) could be taken as a definition. \\
Another useful lemma for locally noetherian schemes is the following:
\begin{lemma}\cite[Stacks, 29.9.5]{stacks-project}\label{locally noetherian 2}
Let $X$ be a locally noetherian scheme. Let $\mathcal{F}, \mathcal{G}$ be coherent $\mathcal{O}_X$-modules. Let $\phi : \mathcal{F} \to \mathcal{G}$ be a morphism and $x \in X$. \\
(i) If $\mathcal{F}_x=0$ then there is an open neighbourhood $U \subset X$ of $x$ such that $\mathcal{F}|_U=0$, \\
(ii) if $\phi_x:\mathcal{F}_x \to \mathcal{G}_x$ is injective, then there is an open neighbourhood $U \subset X$ of $x$ such that $\phi|_U$ is injective, \\
(iii) if $\phi_x:\mathcal{F}_x \to \mathcal{G}_x$ is surjective, then there is an open neighbourhood $U \subset X$ of $x$ such that $\phi|_U$ is surjective, \\
(iv) if $\phi_x : \mathcal{F}_x \to \mathcal{G}_x$ is bijective, then there is an open neighbourhood $U \subset X$ of $x$ such that $\phi|_U$ is an isomorphism.
\end{lemma}
\begin{lemma}\label{coerente livre}
A coherent sheaf $\mathcal{F}$ on a locally noetherian scheme $X$ with free stalks is a locally free sheaf.
\end{lemma}
\begin{proof}
Being locally free is a local concept, so assume $X=Spec(A)$ and $\mathcal{F}=\bar{M}$. By Lemma (\ref{locally noetherian}) $M$ is a finitely generated module over $A$. Then $\bar{M}$ is locally free if and only if $M$ is projective. But a finitely generated module $M$ is projective if and only if all its localizations $M_\mathfrak{p}$ are $A_\mathfrak{p}$ free modules.
\end{proof}
From now on every scheme $X$ will be locally noetherian, Lemma (\ref{coerente livre}) gives a proposition which will be frequently used:
\begin{proposition}\label{kernel locally free}
Let $\beta : \mathcal{F} \to \mathcal{G}$ be a surjective morphism of locally free sheaves, then $ker \beta$ is a locally free sheaf.
\end{proposition}
\begin{proof}
Let $x \in X$. $\mathcal{G}_x$ is free since $\mathcal{G}$ is locally free. In particular it is a projective module. Then the sequence of modules
\begin{equation}
0 \to ker \beta_x \to \mathcal{F}_x \to \mathcal{G}_x \to 0
\end{equation}
splits. Hence $\mathcal{F}_x=ker \beta_x \oplus \mathcal{G}_x$, and $ker \beta_x$ is projective since $\mathcal{F}_x$ is free. A projective module over a local ring is free which implies that $ker \beta_x$ is free.
The kernel of a morphism between coherent sheaves on a locally noetherian scheme is also coherent, and a coherent sheaf with free stalks is locally free by Lemma (\ref{coerente livre}).
\end{proof}
\begin{definition}
	Let $\mathcal{F}$ be a coherent sheaf on a projective variety $X$. We define its support to be the closed set $Supp(\mathcal{F}) = \{ x \in X | \ \mathcal{F}_x \not= 0 \}$ where $\mathcal{F}_x$ denotes the stalk of $\mathcal{F}$ at $x$. The singular locus of $\mathcal{F}$ is the closed set $Sing(\mathcal{F}) = \{x \in X | \ \mathcal{F}_x 	 \ is \ not \ a \ free \ \mathcal{O}_{X,x}-mod \}$.
\end{definition}
The next theorem tells us we can look at the sheaves  $\mathcal{E}xt^p(\mathcal{F},\mathcal{O}_X)$ to see where $\mathcal{F}$ fails to be free
\begin{theorem} The singular locus of $\mathcal{F}$ is
	$\bigcup\limits_{p=1}^{dimX} Supp \ \mathcal{E}xt^p(\mathcal{F}, \mathcal{O}_X)$
\end{theorem}

\begin{proof}
	\cite[Okonek, Schneider \& Spindler, Lemma 1.1.4]{Okonek}
\end{proof}
We are going to introduce some essential concepts of nonsingular varieties. They are going to be extensively used through the rest of this dissertation.
\begin{definition}[Sheaf of relative differentials]
Let $f:X \to Y$ be a morphism of schemes. Consider the diagonal morphism $\Delta: X \to X \times_Y X$, the diagonal morphism gives an isomorphism between $X$ and $\Delta(X)$ so we can consider $\Delta(X)$ as a locally closed subscheme of $X \times_Y X$, i.e. a closed subscheme of an open set of $X \times_Y X$. Let $\mathcal{I}$ be the ideal sheaf of $\Delta(X)$. We define the sheaf of relative differentials of $X$ over $Y$ to be the sheaf $\Omega_{X/Y}=\Delta^* (\mathcal{I}/\mathcal{I}^2)$ on $X$. It is a quasi-coherent sheaf of $\mathcal{O}_X$-modules and if $Y$ is noetherian and $f$ is of finite type then $\Omega_{X/Y}$ is coherent.
\end{definition}
\begin{proposition}[The Euler sequence]
Let $A$ be a ring, let $Y=Spec(A)$ and $X=\mathbb{P}^n_A$. Then there is an exact sequence of sheaves on $X$.
\begin{equation}
0 \to \Omega_{X/Y} \to \mathcal{O}_X(-1)^{n+1} \to \mathcal{O}_X \to 0
\end{equation}
\end{proposition}
\begin{proof}
\cite[Hartshorne, Theorem II.8.13]{hartshorne_2010}
\end{proof}
If $X$ is a nonsingular variety of dimension $n$ over $k$ we define the \textbf{tangent sheaf} $\mathcal{T}_X=\mathcal{H}om_{\mathcal{O}_X}(\Omega_{X/k},\mathcal{O}_X)$ and the \textbf{canonical sheaf} $\omega_X=\bigwedge^n \Omega_{X/k}$. The tangent sheaf is locally free of rank $n$ and the canonical sheaf is invertible. If $Y \overset{i}{\to}X$ is a nonsingular subvariety we define the \textbf{conormal sheaf} of $Y$ on $X$ as $\mathcal{C}_{Y/X}=i^*(\mathcal{I}/\mathcal{I}^2)$ where $\mathcal{I}$ is the ideal sheaf of $Y$ in $X$. Its dual $\mathcal{N}_{Y/X}=\mathcal{H}om_{\mathcal{O}_Y}(\mathcal{C}_{Y,X},\mathcal{O}_Y)$ is the \textbf{normal sheaf} of $Y$ in $X$. Both are locally free.\\
We need one more theorem from \cite[Hartshorne]{hartshorne_2010}:
\begin{theorem}\label{nonsing}
Let $X$ be a nonsingular variety over $k$ and $Y\subset X$ an irreducible closed subscheme defined by a sheaf of ideals $\mathcal{I}$. Then $Y$ is nonsingular if and only if\\
(i) $\Omega_{Y/X}$ is locally free and\\
(ii) the sequence $0 \to \mathcal{I}/\mathcal{I}^2 \to \Omega_{X/k} \otimes \mathcal{O}_Y \to \Omega_{Y/k} \to 0$ is exact
\end{theorem}
\begin{proof}
\cite[Hartshorne, theorem II.8.17]{hartshorne_2010}
\end{proof}
\begin{proposition}\label{canonico}
Let $Y$ be a nonsingular variety of codimension $r$ in a nonsingular variety $X$ over $k$. Then $\omega_Y \cong \omega_X|_Y \otimes \bigwedge^r \mathcal{N}_{Y/X}.$
\end{proposition}
\begin{proof}
Since $Y$ is nonsingular \ref{nonsing} gives an exact sequence
\begin{equation}
0 \to \mathcal{I}/\mathcal{I}^2 \to \Omega_{X/k} \otimes \mathcal{O}_Y \to \Omega_{Y/k} \to 0
\end{equation}
Taking the highest exterior powers gives $\omega_X \otimes \mathcal{O}_Y \cong \omega_Y \otimes \bigwedge^r(\mathcal{I}/\mathcal{I}^2)$. Applying $\otimes \mathcal{N}_{Y/X}$ and commuting the highest exterior power with the dual yields $\omega_Y \cong \omega_X \otimes \bigwedge^r \mathcal{N}_{Y/X}$.
\end{proof}
Proposition \ref{canonico} gives a way to compute the canonical sheaf of nonsingular varieties:
\begin{example}[The canonical sheaf of a hypersurface]\label{canonicosurface}
Let $X=\mathbb{P}^n_k$ for $n \geq 2$ and $Y$ be a nonsingular hypersurface of degree $d$. Taking the highest exterior power from the Euler sequence gives $\omega_X \cong \mathcal{O}_X(-n-1)$, so by (\ref{canonico}) $\omega_Y \cong \mathcal{O}_X(-n-1) \otimes \mathcal{N}_{Y/X}$. Since $Y$ is a hypersurface of degree $d$ we have the exact sequence $0 \to \mathcal{O}_X(-d) \to \mathcal{O}_X \to \mathcal{O}_Y \to 0$ so $\mathcal{I}/\mathcal{I}^2 \cong \mathcal{I} \otimes \mathcal{O}_Y \cong \mathcal{O}(-d)|_Y$ and hence $\mathcal{N}_{Y/X} \cong \mathcal{O}_Y(d)$ which implies $\omega_Y \cong \mathcal{O}_Y(d-n-1)$.
\end{example}
\section{Curves and Chern classes}
In this section a curve will be an integral scheme of dimension 1, proper over $k$ with regular local rings.
\begin{definition}
	Let $X$ be a projective scheme of dimension $r$ over $k$ and $\mathcal{F}$ be a coherent sheaf on $X$. The Euler characteristic of $\mathcal{F}$ is defined by
	\begin{equation}
		\chi^e(\mathcal{F}) = \Sigma_{i=0}^{r} (-1)^i h^i(X,\mathcal{F})
	\end{equation}
Here $h^i(X,\mathcal{F})$ denotes the dimension of $H^i(X,\mathcal{F})$ as a $k$-vector space
\end{definition}
Note that if $0 \to \mathcal{F}_1 \to \mathcal{F}_2 \to \mathcal{F}_3 \to 0$ is an exact sequence of coherent sheaves on $X$, then we have an exact sequence of finite dimensional $k$-vector spaces
\begin{equation}
0 \to H^0(X,\mathcal{F}_1) \to H^0(X,\mathcal{F}_2) \to H^0(X,\mathcal{F}_3) \to H^1(X,\mathcal{F}_1) \to ...
\end{equation}
which implies $\chi^e(\mathcal{F}_2)=\chi^e(\mathcal{F}_1)+\chi^e(\mathcal{F}_3)$. In other words, $\chi^e$ is additive.
\begin{definition}
	Let $X$ be a projective scheme of dimension $r$ over $k$. We define the arithmetic genus $p_a(X)$ of $X$ by:
	\begin{equation}
		p_a(X)=(-1)^r(\chi^e(\mathcal{O}_X)-1)
	\end{equation}
	If $X$ is a curve we also define the geometric genus $p_g(X)=h^0(X,\omega_X)$, where $\omega_X$ is the canonical sheaf.
\end{definition}
If $X$ is a curve then $h^r(\mathcal{O}_X)=0$ for $r \geq 2$ so $p_a(X)=(-1)(h^0(\mathcal{O}_X)-h^1(\mathcal{O}_X)-1)$. Since $X$ is integral it corresponds to a projective variety, hence $H^0(\mathcal{O}_X)=k$ which implies $p_a(X)=h^1(\mathcal{O}_X)$. Serre duality implies $H^0(X,\omega_X) \cong Ext^1(\omega_X,\omega_X)^* \cong Ext^1(\mathcal{O}_X,\mathcal{O}_X)^* \cong H^1(X,\mathcal{O}_X)^*$ because $\omega_X$ is invertible, so $h^0(X,\omega_X)=h^1(X,\mathcal{O}_X)$. This leads to the definition of an important number
\begin{definition}
	Let $X$ be a curve, then $p_a(X)=p_g(X)=h^1(X,\mathcal{O}_X)$ is called the genus of $X$. We denote it by $g$.
\end{definition}
Since we have only defined those numbers for projective schemes, it is important to notice that a curve $X$ is necessarily projective since $X$ is proper over $k$. \cite[Hartshorne II.6.7]{hartshorne_2010}.\\
We will briefly recall some basic definitions on divisors.\\
Let $X$ a notherian integral separated scheme which is regular on codimension one \cite[Hartshorne, p.130]{hartshorne_2010}. A Weil divisor on $X$ is an element of the free abelian group $Div(X)$ generated by prime divisors, in other words generated by closed integral subscheme $Y$ of $X$. We can write a divisor $D=\Sigma_{i=1}^r n_i Y_i$ with $n_i \in \mathbb{Z}$. The degree of the divisor $D$ is defined to be $deg(D)=\Sigma_{i=1}^r n_i$. A divisor $D$ is called effective if $n_i \geq 0$ for every $i$.\\
Since $X$ is integral it has a generic point $\xi \in X$. The function field of an integral scheme is defined as the stalk at its generic point $K(X)=\mathcal{O}_{X,\xi}$. An element of $K(X)$ is called a rational function. A prime divisor $Y$ is integral so we can consider the stalk at its generic point $\mathcal{O}_{X,\eta}$ which has quotient field $K(X)$ and it is a discrete valuation ring (this comes from the fact that a ring $A$ is a discrete valuation ring if and only if it is noetherian, normal and it has only two prime ideals : $0$ and $m$). So it makes sense to talk about the evaluation $v_Y(f)$ of $f \in K(X)^*=K(X)- \{ 0 \} $ at $Y$. We define the divisor of $f$ to be $div(f)=\Sigma v_Y(f)Y$, where $\Sigma$ runs over all prime divisors. It is well defined because
any proper closed subset of $X$ can contain only a finite amount of prime divisors. \cite[Hartshorne, II.6.1]{hartshorne_2010}\\
Given two divisors $D,D'$ we say $D$ is linearly equivalent to $D'$ if $D-D'=div(f)$ for some rational function $f$. The quotient group $Div(X)$ modulo linear equivalence is called the divisor class group $Cl(X)$ and it is isomorphic to the group $Pic(X)$ of invertible sheaves on $X$ modulo isomorphism \cite[Hartshorne, II.6.10]{hartshorne_2010}, the image of $[D] \in Div(X)/ \sim$ in $Pic(X)/ \sim$ is denoted $\mathcal{L}(D)$. A divisor $K$ corresponding to the invertible sheaf $\omega_X$ is said to be a canonical divisor.

The Riemann-Roch theorem was originally proven by Riemann in 1857 \cite[Riemann]{Riemann} and later generalized by Roch in 1865 \cite[Roch]{Roch}, it was first developed in an analytic context. We present here a version of Riemann-Roch concerning curves in algebraic geometry.
\begin{theorem}\cite[Hartshorne, Riemann-Roch Theorem p.295]{hartshorne_2010}\label{riemannrochcurve}
	Let $D$ be a divisor on a curve $X$ of genus $g$. Then
	\begin{equation}
	h^0(\mathcal{L}(D))-h^0(\mathcal{L}(K-D))=deg(D)+1-g
	\end{equation}
	where $K$ is a canonical divisor.
\end{theorem}
Theorem (\ref{riemannrochcurve}) is an important tool used to compute cohomology of invertible sheaves. The divisor $K-D$ corresponds to an invertible sheaf $\omega_X \otimes \mathcal{L}(D)^*$, using Serre duality we conclude $H^0(\omega_X \otimes \mathcal{L}(D)^*) \cong H^1(\mathcal{L}(D))^*$. Thus the Riemann-Roch theorem simply states $\chi^e(\mathcal{L}(D))=deg(D)+1-g$.

\begin{example}
	Let $Y \overset{i}{\to} X=\mathbb{P}^3$ be a conic. It is of the form $V(f_1,f_2)$ where $f_1 \in H^0(\mathcal{O}_{\mathbb{P}^3}(1))$ and $f_2 \in H^0(\mathcal{O}_{\mathbb{P}^3}(2))$ and it has degree $deg(f_1)deg(f_2)=2$. The canonical sheaf $\omega_Y$ is isomorphic to $\mathcal{O}_Y(-1)$ by Proposition (\ref{canonicosurface}) thus $K$ is linearly equivalent to $-L$ where $L$ is the divisor corresponding to $i^*\mathcal{O}_X(1)$, it has degree $deg(L)=2$ because the intersection of a general hyperplane with $Y$ has two points. Apply Theorem (\ref{riemannrochcurve}) we get: $h^0(\omega_Y)-h^0(\mathcal{O}_Y)=deg(K)+1-g$. By definition $h^0(\omega_Y)=g$ so $g=(-deg(H)+2)/2=0$. Now let $r \in \mathbb{Z}$ and let $D=rL$. $\mathcal{L}(D)=\mathcal{O}_Y(r)$ so $h^0(\mathcal{O}_Y(r))-h^1(\mathcal{O}_Y(r))=1+2r$.
\end{example}
We won't use theorem (\ref{riemannrochcurve}), but we will use one of the many generalizations, the Hirzebruch-Riemann-Roch. But first we need a few definitions.
\begin{definition}(Chern class of a vector bundle)\cite[Hartshorne, p.429]{hartshorne_2010}\label{defchern}
Let $\mathcal{E}$ be a locally free sheaf of rank $r$ on a nonsingular quasi-projecive variety $X$. For each $i=0,...,r$ we define the $i$th Chern class $c_i(\mathcal{E})\in A^i(X)$ ($A^i(X)$ is the ith Chow group) by the requirement $c_0(\mathcal{E})=1$ and
\begin{equation}
\Sigma_{i=0}^r (-1)^i \pi^* c_i(\mathcal{E}).\xi^{r-i}=0
\end{equation}
in $ A^r(\mathbb{P}(\mathcal{E}))$ where $\pi : \mathbb{P}(\mathcal{E}) \to X$ is the projection from the associated projective bundle of $\mathcal{E}$ and $\xi \in A^1(\mathbb{P}(\mathcal{E}))$ is the class of the divisor corresponding to $\mathcal{O}_{\mathbb{P}(\mathcal{E})}(1)$.
\end{definition}
We will be most interested in the case $X=\mathbb{P}^n$. In this case $A^i(X)\cong\mathbb{Z}$ is generated by $H^d$ where $H$ is the hyperplane class associated to $\mathcal{O}_{\mathbb{P}^n}(1)$. So $A(X)=\bigoplus_{i \geq 0}A^i(X)$ is just $\mathbb{Z}[H]/H^{n+1}$ as a group since $A^{n+1}(X)=0$. It is possible to define a product in $A(X)$, making it isomorphic to $\mathbb{Z}[H]/H^{n+1}$ as a ring. The product $A^d(X) \times A^e(X) \to A^{d+e}(X)$ can be seen as the general intersection of varieties. So we can identify $c_i(\mathcal{E})$ with an integer for every $i$ and make computations with them as if they were in $\mathbb{Z}[H]/H^{n+1}$. The Chern polynomial $c_t(\mathcal{E})$ is just $1+c_1(\mathcal{E})t+...+c_r(\mathcal{E})t^r$. The Chern polynomial is multiplicative in the sense that if $0 \to \mathcal{E}' \to \mathcal{E} \to \mathcal{E}'' \to 0$ is exact then $c_t(\mathcal{E})=c_t(\mathcal{E}')c_t(\mathcal{E}'')$

At some point we will be talking about the Chern class of a coherent sheaf, this is possible because since $c_t$ is multiplicative, it can be defined on the Grothendieck group of vector bundles on $X$ \cite[Hartshorne, p.435]{hartshorne_2010} and for a nonsingular variety $X$ the Grothendieck groups of all vector bundles and of all coherent sheaves coincide \cite[Hartshorne, III Ex. 6.9]{hartshorne_2010}. Using this isomorphism we can define the Chern class of a coherent sheaf on $X$.

Let $c_t(\mathcal{E})=\prod_{i=1}^r (1+a_iH)$ considering $c_t$ in $A(X) \otimes \mathbb{Q}$ where $a_i$ are formal symbols, the exponential Chern character of $\mathcal{E}$ is $ch(\mathcal{E})=\Sigma_{i=1}^r e^{a_i}$ where $e^x=1+x+\frac{x^2}{2}+...$  and the Todd class is defined to be $td(\mathcal{E})=\prod_{i=1}^{r} \frac{a_i}{1-e^{-a_i}}$ where $\frac{x}{1-e^{-x}}=1+\frac{x}{2}+\frac{x^2}{12}-\frac{x^4}{720}+...$.
\begin{theorem}(Hirzebruch-Riemann-Roch)\label{hirzebruchriemannroch}
	For a locally free sheaf $\mathcal{E}$ of rank $r$ on a nonsingular projective variety $X$ of dimension $n$,
	\begin{equation}
	\chi^e(\mathcal{E})=deg(ch(\mathcal{E}).td(\mathcal{T}_X))_n
	\end{equation}
	where $()_n$ denotes the component of degree $n$ in $A(X)\otimes \mathbb{Q}$
\end{theorem}
The Hirzebruch-Riemann-Roch theorem will be used in chapter 3 to develop the formula (\ref{riemannroch}).

Let $\mathcal{E}$ be a vector bundle of rank 2 on $\mathbb{P}^3$. We have the following formulas: \\
(i) $c_1(\mathcal{E}(r))=c_1(\mathcal{E})+2r$ \\
(ii) $c_2(\mathcal{E}(r))=c_2(\mathcal{E})+c_1(\mathcal{E})r+r^2$ \\
(iii) $c_i(\mathcal{E}^*)=(-1)^i c_i(\mathcal{E})$\\
If $\mathcal{E}$ has first Chern class $c_1 \in \{-1,0\}$ we say $\mathcal{E}$ is \textbf{normalized}.
Furthermore we can always normalize a vector bundle $\mathcal{E}$, in other words we can always find a $r \in \mathbb{Z}$ so that $c_1(\mathcal{E}(r)) \in \{-1,0 \}$.

We end this section with the notion of stability of torsion-free sheaves
\begin{definition}[Stability of torsion-free sheaves]
	Let $\mathcal{F}$ be a torsion-free coherent sheaf on $X$, a normal projective variety with fixed very ample divisor $H$. We define the slope of $\mathcal{F}$ to be $\mu({\mathcal{F}}):=deg(c_1(\mathcal{F}))/rk(\mathcal{F})$ where we are considering $c_1$ as an element of $Pic(X)$ and $deg$ is the degree with respect to $H$. $\mathcal{F}$ is \textbf{stable} if and only if $\mu({\mathcal{H}})<\mu({\mathcal{F}})$ for all proper non-zero coherent subsheaves $\mathcal{H}$ of $\mathcal{F}$ (and is \textbf{semistable} if $\mu({\mathcal{H}}) \leq \mu({\mathcal{F}})$).
\end{definition}
A normalized vector bundle $\mathcal{E}$ of rank 2 on $\mathbb{P}^3$ is stable if and only if $H^0(\mathcal{E})=0$ \cite[Okonek, Schneider \& Spindler, p. 167]{Okonek}.
\section{Monads}
In this section we will define one of the main objects of this dissertation.
\begin{definition}\label{def monad}
	A monad $\mathcal{V}_\bullet$ on $X$ is a complex of locally free sheaves on X
	\begin{equation}
	\mathcal{V}_\bullet : \mathcal{V}_0 \overset{ \alpha }{ \to} \mathcal{V}_1 \overset{ \beta }{ \to} \mathcal{V}_2
	\end{equation}
	such that $\alpha$ is injective and $\beta$ is surjective as morphisms of sheaves. The sheaf $\mathcal{E}=ker \beta / im \alpha$ is called the cohomology of $\mathcal{V}_\bullet$.  \\
	The degeneration locus of $\mathcal{V}_\bullet$ is the set $\Sigma_{\mathcal{V}_\bullet} =\{x \in X | \ \alpha(x) \ is \ not \ injective  \} $
\end{definition}
Monads were introduced by Horrocks, who proved that every vector bundle on $\mathbb{P}^n$ is the cohomology of a minimal monad (\cite[Horrocks]{doi:10.1112/plms/s3-14.4.689} or \cite[Barth]{Barth1978}). In fact Horrocks shows that every rank 2 vector bundle on $\mathbb{P}^3$ is the cohomology of a monad $\mathcal{V}_0 \overset{ \alpha }{ \to} \mathcal{V}_1 \overset{ \beta }{ \to} \mathcal{V}_2$ where $\mathcal{V}_i$ are sums of line bundles. We are only interested in this particular situation. But before moving on we need an important theorem given by Beilinson.
\begin{theorem}\cite[Okonek, Schneider \& Spindler, p. 240]{Okonek}\label{beilinson1}
For any locally free sheaf $\mathcal{E}$ on $\mathbb{P}^n$ there exists a spectral sequence {$\mathcal{E}^{p,q}_r$} for $\mathcal{E}$ whose $\mathcal{E}_1$-term is given by ($q=0,...,n$ and $p=0,-1,...,-n$):
\begin{equation}
\mathcal{E}^{p,q}_1=H^q(\mathcal{E}(p)) \otimes \Omega^{-p}_{\mathbb{P}^n}(-p)
\end{equation}
which converges to 
\begin{equation}
\mathcal{E}^{i} = \begin{cases} \mathcal{E}, if \ p+q=0 \\ 0 \ otherwise
\end{cases}
\end{equation}
\end{theorem}
\begin{theorem}\cite[Okonek, Schneider \& Spindler, p. 245]{Okonek}\label{beilinson2}
For any locally free sheaf $\mathcal{E}$ on $\mathbb{P}^n$ there exists a spectral sequence {$\mathcal{E}^{p,q}_r$} for $\mathcal{E}$ whose $\mathcal{E}_1$-term is given by ($q=0,...,n$ and $p=0,-1,...,-n$):
\begin{equation}
\mathcal{E}^{p,q}_1=H^q(\mathcal{E}\otimes\Omega^{-p}_{\mathbb{P}^n}(-p)) \otimes \mathcal{O}_{\mathbb{P}^n}(p)
\end{equation}
which converges to 
\begin{equation}
\mathcal{E}^{i} = \begin{cases} \mathcal{E}, if \ p+q=0 \\ 0 \ otherwise
\end{cases}
\end{equation}
\end{theorem}
In addition with theorems such as the Serre duality and Riemann-Roch the Beilinson spectral sequence is an important tool for characterizing vector bundles on $\mathbb{P}^n$.

We conclude the chapter giving an example on how to use the Beilinson spectral sequence. This result will also be used in chapter 2.
\begin{lemma}\label{lemabeilinson}
Let $\mathcal{F}$ be a coherent sheaf on $\mathbb{P}^n$ and $\{ \mathcal{F}^{p,q}_r \}$ its Beilinson spectral sequence. If $\mathcal{F}^{p,q}_1=0$ for $q \not=1$ and for $q=1$, $p \leq -3$ then the spectral sequence degenerates at the $\mathcal{F}_2$-term and the monad
\begin{equation}
0 \to \mathcal{F}^{-2,1}_\infty \to \mathcal{F}^{-1,1}_\infty \to \mathcal{F}^{0,1}_\infty \to 0
\end{equation}
has $\mathcal{F}$ as its cohomology.
\end{lemma}
\begin{proof}
We have the following diagram on $\mathcal{F}_1$ : \\
\begin{tikzpicture}
  \matrix (m) [matrix of math nodes,
    nodes in empty cells,nodes={minimum width=5ex,
    minimum height=5ex,outer sep=-5pt},
    column sep=2ex,row sep=3ex]{    
     &          &            &           &  q    &   \\
     &     0    &  0         &  0        &   0   &   \\
     &     0    &\mathcal{F}^{-2,1}_1  & \mathcal{F}^{-1,1}_1 &   \mathcal{F}^{0,1}_1  &   \\
     &     0    &  0         &  0        &   0   &  p \\ 
  \quad\strut  &  &    &   &   &  \strut \\};
  \draw[-stealth] (m-2-2) -- (m-2-3);
  \draw[-stealth] (m-2-3) -- (m-2-4);
  \draw[-stealth] (m-2-4) -- (m-2-5);
  \draw[-stealth] (m-3-2) -- (m-3-3);
  \draw[-stealth] (m-3-3) -- (m-3-4);
  \draw[-stealth] (m-3-4) -- (m-3-5);
  \draw[-stealth] (m-4-2) -- (m-4-3);
  \draw[-stealth] (m-4-3) -- (m-4-4);
  \draw[-stealth] (m-4-4) -- (m-4-5);
\draw[->]  (m-5-5.east) -- (m-1-5.east) ;
\draw[->] (m-4-1.south) -- (m-4-6.south) ;
\end{tikzpicture} \\
The differentials $d^{pq}_1: \mathcal{F}^{p,q}_1 \to \mathcal{F}^{p+1,q}_1$ give the complex 
\begin{equation}
\mathcal{F}^{-2,1}_1 \overset{\alpha}{\to} \mathcal{F}^{-1,1}_1 \overset{\beta}{\to} \mathcal{F}^{0,1}_1
\end{equation}
Let $\mathcal{K}=ker \ \alpha$, $\mathcal{L}=ker \ \beta / im \ \alpha$, $\mathcal{M}=coker \ \beta$, then the diagram of the $\mathcal{F}_2$-term looks as follows :

\begin{tikzpicture}
  \matrix (m) [matrix of math nodes,
    nodes in empty cells,nodes={minimum width=5ex,
    minimum height=5ex,outer sep=-5pt},
    column sep=2ex,row sep=3ex]{    
     &          &            &           &  q    &   \\
     &     0    &  0         &  0        &   0   &   \\
          &     0    &\mathcal{K}  & \mathcal{L} &   \mathcal{M}  &   \\ 
     &     0    &  0         &  0        &   0   &   \\
  \quad\strut  &  &    &   &   & p \strut \\};
  \draw[-stealth] (m-2-2) -- (m-3-3);
  \draw[-stealth] (m-3-3) -- (m-4-4);
  \draw[-stealth] (m-2-3) -- (m-3-4);
  \draw[-stealth] (m-3-4) -- (m-4-5);
  \draw[-stealth] (m-2-4) -- (m-3-5);
\draw[->]  (m-5-5.east) -- (m-1-5.east) ;
\draw[->] (m-4-1.south) -- (m-4-6.south) ;
\end{tikzpicture} \\
All the differentials $d^{pq}_2 : \mathcal{F}^{p,q}_2 \to \mathcal{F}^{p+2,q-1}_2$ vanish and so  $\mathcal{K}=\mathcal{F}^{-2,1}_\infty$, $\mathcal{L}= \mathcal{F}^{-1,1}_\infty$, $\mathcal{M}=\mathcal{F}^{0,1}_\infty$.
But by Beilinson's theorem $\mathcal{K}=\mathcal{M}=0$ and $\mathcal{L}=\mathcal{F}$. In particular 
\begin{equation}
0 \to \mathcal{F}^{-2,1}_\infty \to \mathcal{F}^{-1,1}_\infty \to \mathcal{F}^{0,1}_\infty \to 0
\end{equation}
is a monad whose cohomology is $\mathcal{F}$.
\end{proof}
\chapter{Linear and Horrocks monads}
%\chapter{An introduction to the study of monads}
This chapter contains the main goal of the first year of the master's degree: to study and understand the demonstrations from the first sections of \cite[Jardim]{jardim} and \cite[Jardim \& Martins]{jardim2} serving as an introduction to the study of monads, vector bundles and cohomology.

In the first section of \cite[Jardim]{jardim} it is studied the cohomology of linear monads, including a theorem that guarantees a given torsion-free sheaf to be linear provided some conditions on its cohomology.\\
In the first two sections of \cite[Jardim \& Martins]{jardim2} it is given a bijective correspondence between isomorphism classes of monads and collections of homogeneous elements of $H^1_*(\mathcal{E})$ and $H^1_*(\mathcal{E}\otimes\omega_X)$ where $\mathcal{E}$ is locally free and $X$ is an ACM variety. It is also studied the cohomology functor that associates to each Horrocks monad its cohomology sheaf.
\section{Linear monads}
Let $\mathcal{V}_\bullet: \mathcal{V}_0 \overset{\alpha}{\to} \mathcal{V}_1 \overset{\beta}{\to} \mathcal{V}_2$ be a monad, it is worth noticing that its cohomology $\mathcal{E}$ is not always locally free, but conditions on $\alpha$ can be set to guarantee it.
Next proposition states that $\mathcal{E}$ is locally free if and only if $\alpha (x)$ is injective for every $x$.
\begin{proposition}
The degeneration locus of $\mathcal{V}_\bullet$ coincides with the singular locus of its cohomology $\mathcal{E}$. In other words $\Sigma_{\mathcal{V}_\bullet} = Sing(\mathcal{E}) = Supp(\mathcal{E}xt^1(\mathcal{E},\mathcal{O}_X))$
\end{proposition}
\begin{proof}
Consider the short exact sequence :
\begin{equation}
0 \to \mathcal{V}_0 \to \mathcal{K} \to \mathcal{E} \to 0
\end{equation}
where $\mathcal{K} = ker \beta$. Applying the functor $\mathcal{H}om(-,\mathcal{O}_{\mathbb{P}^n})$ the sequence yields:
\begin{equation}
0 \to \mathcal{E}^* \to \mathcal{K}^* \overset{\alpha^*}{\to} \mathcal{V}_0^* \to \mathcal{E}xt^1(\mathcal{E},\mathcal{O}_X) \to 0
\end{equation}
$\mathcal{K}$ being locally free by \ref{kernel locally free}, and $\mathcal{V}_0$ is locally free by hypothesis which implies $\mathcal{E}xt^p(\mathcal{E},\mathcal{O}_X)=0$ for $p \geq 2$. Thus $Sing(\mathcal{E}) = Supp(\mathcal{E}xt^1(\mathcal{E}, \mathcal{O}_X))$. \\
The first equality follows from the fact that surjectivity on the stalks happens if and only if the morphism of fibers is surjective. \\
Then $\alpha (x)$ is injective if and only if $\alpha^* (x)$ is surjective if and only if $\alpha^*_x$ is surjective if and only if $x \not\in \Sigma_{V\bullet}$.
\end{proof}

In 1977 \cite[Atiyah, Hitchin, Drinfeld \& Manin]{ATIYAH1978185} introduced the notion of a mathematical instanton bundle on $\mathbb{P}^3$. Instantons are self-dual solutions of the Yang-Mills equations in the compactified euclidean 4-space $S^4$ which corresponds to certain real algebraic bundles on $\mathbb{P}^3_{\mathbb{C}}$.
In 1986 \cite[Okonek \& Spindler]{Spindler} generalized this idea to define mathematical instanton bundles on $\mathbb{P}^{2n+1}$.

Each instanton bundle appears as the cohomology of a monad of the form
\begin{equation}
\mathcal{O}_{\mathbb{P}^{2n+1}}(-1)^r \overset{ \alpha}{\to} \mathcal{O}_{\mathbb{P}^{2n+1}}^{2n+2r} \overset{ \beta}{\to}  \mathcal{O}_{\mathbb{P}^{2n+1}}(1)^r
\end{equation}
This motivates the following definition
\begin{definition}\cite[Jardim]{jardim}
A monad is said to be linear if it is of the form :
\begin{equation}
V \otimes \mathcal{O}_{\mathbb{P}^n}(-1) \overset{ \alpha}{\to} W \otimes \mathcal{O}_{\mathbb{P}^n} \overset{ \beta}{\to} U \otimes \mathcal{O}_{\mathbb{P}^n}(1)
\end{equation}
Where $V,W,U$ are finite dimensional vector spaces, and the product $V \otimes \mathcal{O}_X$ denotes the locally free sheaf of rank $dim(V)$ with fiber $V$. \\ 
A coherent sheaf will be called linear if it can be represented as the cohomology of a linear monad.
\end{definition}
The next theorem will give some necessary conditions for a sheaf $\mathcal{E}$ to be linear.

\begin{theorem}\label{jardim1}\cite[Jardim]{jardim}
If $\mathcal{E}$ is a linear sheaf, then: \\
(i) for $n \geq 2$, $H^0(\mathcal{E}(k))=H^0(\mathcal{E}^*(k))=0$, $\forall k \leq -1$ \\
(ii) for $n \geq 3$, $H^1(\mathcal{E}(k))=0$, $\forall k \leq -2$ \\
(iii) for $n \geq 4$, $H^p(\mathcal{E}(k))=0$, $2 \leq p \leq n-2$ and $\forall k$ \\
(iv) for $n \geq 3$, $H^{n-1}(\mathcal{E}(k))=0$, $\forall k \geq -n+1$ \\
(v) for $n \geq 2$, $H^n(\mathcal{E}(k))=0$ for $k \geq -n$ \\
(vi) for $n \geq 2$, $\mathcal{E}xt^1(\mathcal{E},\mathcal{O}_{P^n})=coker \alpha^*$ and $\mathcal{E}xt^p(\mathcal{E}(k),\mathcal{O}_{P^n})=0$ for $p \geq 2$ and all k
\end{theorem}

\begin{proof}
Since $\mathcal{E}$ is a linear sheaf, we can write it as the cohomology of a monad:
\begin{equation}
V \otimes \mathcal{O}_{\mathbb{P}^n}(-1) \overset{ \alpha}{\to} W \otimes \mathcal{O}_{\mathbb{P}^n} \overset{ \beta}{\to} U \otimes \mathcal{O}_{\mathbb{P}^n}(1)
\end{equation}
First note that kernel $\mathcal{K}=ker \beta$ is locally free (\ref{kernel locally free}), and there are two sequences for every integer $k$:
\begin{equation}
0 \to \mathcal{K}(k) \to W \otimes \mathcal{O}_{\mathbb{P}^n}(k) \to U \otimes \mathcal{O}_{\mathbb{P}^n}(k+1) \to 0
\end{equation}
and
\begin{equation}
0 \to V \otimes \mathcal{O}_{\mathbb{P}^n}(k-1) \to \mathcal{K}(k) \to \mathcal{E}(k) \to 0
\end{equation}

From the first sequence writing the long exact sequence of cohomology:
\begin{equation}
0 \to H^0(\mathcal{K}(k)) \to H^0(W \otimes \mathcal{O}_{\mathbb{P}^n}(k)) \to H^0(U \otimes \mathcal{O}_{\mathbb{P}^n}(k+1)) \to ...
\end{equation}
If $k \leq -1$ then $H^0(W \otimes \mathcal{O}_{\mathbb{P}^n}(k))=0$ so that $H^0(\mathcal{K}(k))=0$.
Continuing the long sequence:
\begin{equation}
H^0(U \otimes \mathcal{O}_{\mathbb{P}^n}(k+1)) \to H^1(\mathcal{K}(k)) \to H^1(W \otimes \mathcal{O}_{\mathbb{P}^n}(k)) \to ...
\end{equation} 
If $k \leq -2$ then $H^0(U \otimes \mathcal{O}_{\mathbb{P}^n}(k+1))=0$. Furthermore if $n \geq 2$ then $H^1 (\mathcal{O}_{\mathbb{P}^n}(k))=0$ and therefore $H^1(W \otimes \mathcal{O}_{\mathbb{P}^n}(k))=0$, implying $H^1(\mathcal{K}(k))=0$. \\
Consider now $2 \leq p \leq n-1$ and the sequence
\begin{equation}
H^{p-1}(U \otimes \mathcal{O}_{\mathbb{P}^n}(k+1)) \to H^{p}(\mathcal{K}(k)) \to H^{p}(W \otimes \mathcal{O}_{\mathbb{P}^n}(k))
\end{equation}
If $p-1,p \leq n-1$ then the extremities are $0$, thus $H^p(\mathcal{K}(k))=0$. \\
Finally if $p=n$, $H^p(\mathcal{O}_{\mathbb{P}^n}(k))=(\frac{1}{T_0 \ldots T_n} \mathbb{C}[\frac{1}{T_0}, \ldots, \frac{1}{T_n}])_k$, if $k \geq -n$ this group is $0$ which implies $H^n(\mathcal{K}(k))=0$. \\
Now for the second sequence, again looking at the long exact sequence of cohomology:
\begin{equation}
H^{p}(V \otimes \mathcal{O}_{\mathbb{P}^n}(k-1)) \to H^{p}(\mathcal{K}(k)) \to H^{p}(\mathcal{E}(k)) \to H^{p+1}(V \otimes \mathcal{O}_{\mathbb{P}^n}(k-1))
\end{equation}
The extremities are zero when $p=0$ and $k \leq -1$, $1 \leq p \leq n-2$ $\forall k$, $p=n$ and $k \geq -n$ (because we are on the projective space). Under these conditions we have $H^{p}(\mathcal{K}(k)) \simeq H^{p}(E(k))$. 
All these results gives the first half of (i) through (v). \\
Dualizing both sequences yields:
\begin{equation}
0 \to U \otimes \mathcal{O}_{\mathbb{P}^n}(-k-1) \to W \otimes \mathcal{O}_{\mathbb{P}^n}(-k) \to \mathcal{K}^*(-k) \to 0
\end{equation}
\begin{equation}
0 \to \mathcal{E}^*(-k) \to \mathcal{K}^*(-k) \to V \otimes \mathcal{O}_{\mathbb{P}^n}(-k+1) \to \mathcal{E}xt^1(\mathcal{E}(k),\mathcal{O}_{\mathbb{P}^n}) \to \mathcal{E}xt^1(\mathcal{K}(k),\mathcal{O}_{\mathbb{P}^n}) \to ...
\end{equation}
Writing the long exact sequence for the first sequence :
\begin{equation}
H^0( W \otimes \mathcal{O}_{\mathbb{P}^n}(-k)) \to H^0(\mathcal{K}^*(-k)) \to H^1(U \otimes \mathcal{O}_{\mathbb{P}^n}(-k-1))
\end{equation}
We conclude that $H^0(\mathcal{K}^*(k))=0$ for $k \leq -1$.

For the second sequence, $\mathcal{K}$ locally free implies $\mathcal{E}xt^1(\mathcal{K}(k),\mathcal{O}_{\mathbb{P}^n})=0$. Thus

\begin{equation}
0 \to \mathcal{E}^* \to \mathcal{K}^* \overset{ \alpha^*}{\to} V \otimes \mathcal{O}_{\mathbb{P}^n}(1) \to \mathcal{E}xt^1(\mathcal{E},\mathcal{O}_{\mathbb{P}^n}) \to 0
\end{equation}

So that $coker \alpha^* = \mathcal{E}xt^1(\mathcal{E},\mathcal{O}_{\mathbb{P}^n})$, $H^0(E(k))=0$ for $k \leq -1$ and \\ $\mathcal{E}xt^p(\mathcal{E}(k),\mathcal{O}_{\mathbb{P}^n})=0$ for $p \geq 2$ $\forall k$.

\end{proof}

If $\mathcal{E}$ is a torsion-free sheaf, with some of the conditions on Theorem (\ref{jardim1}) we can guarantee it to be linear. Even further, we can explicitly give its monad:
\begin{theorem}\cite[Jardim]{jardim}
If $\mathcal{E}$ is a torsion-free sheaf on $\mathbb{P}^n$ satisfying: \\
(i) for $n \geq 2$, $H^0(\mathcal{E}(-1))=H^n(\mathcal{E}(-n))=0$; \\
(ii) for $n \geq 3$, $H^1(\mathcal{E}(-2))=H^{n-1}(\mathcal{E}(1-n))=0$; \\
(iii) for $n \geq 4$, $H^p(\mathcal{E}(k))=0$, $2 \leq p \leq n-2$ and $\forall k \in \mathbb{Z}$; \\
then $\mathcal{E}$ is linear and can be represented as the cohomology of the monad:
\begin{equation}
H^1(\mathcal{E} \otimes \Omega^{2}_{\mathbb{P}^n}(1)) \otimes \mathcal{O}_{\mathbb{P}^n}(-1) \to H^1(\mathcal{E} \otimes \Omega^{1}_{\mathbb{P}^n}) \otimes \mathcal{O}_{\mathbb{P}^n} \to H^1(\mathcal{E}(-1)) \otimes \mathcal{O}_{\mathbb{P}^n}(1)
\end{equation}
\end{theorem}

\begin{proof}
Let $H \subset \mathbb{P}^n$ be a hyperplane defined by a homogeneous polynomial $f$ of degree 1.
By a known result \cite[Stacks Project \href{http://stacks.math.columbia.edu/tag/08A0}{Lemma 08A0}]{stacks-project} there is an exact sequence
\begin{equation}
0 \to \mathcal{E} \otimes \mathcal{O}_{\mathbb{P}^n}(k-1) \to \mathcal{E} \otimes \mathcal{O}_{\mathbb{P}^n}(k) \to \mathcal{E} \otimes \mathcal{O}_{\mathbb{P}^n}(k)|_H \to 0
\end{equation}
By hypothesis $H^0(\mathcal{E}(-1))=0$, so for $k=-1$ the long exact sequence yields
\begin{equation}
0 \to H^0(\mathcal{E}(-2)) \to H^0(\mathcal{E}(-1))=0
\end{equation}
And thus $H^0(\mathcal{E}(-2))=0$. Using the same process, by induction results \\ $H^0(\mathcal{E}(k))=0$, $k \leq -1$. \\
For $k=-n+1$ we have
\begin{equation}
0=H^n(\mathcal{E}(-n)) \to H^n(\mathcal{E}(-n+1)) \to H^n(\mathcal{E}(-n+1)|_H)
\end{equation} 
But $H^n(\mathcal{E}(-n+1)|_H)=0$ because $H^n$ vanishes on any Noetherian topological space of dimension $n-1$, and $H^i(\mathbb{P}^{n-1},\mathcal{E}|_H) = H^i(\mathbb{P}^n,i_{*}\mathcal{E}|_H)$ \cite[Hartshorne, Lemma III.2.10]{hartshorne_2010}, hence $H^n(\mathcal{E}(k))=0$ for $k \geq -n$. \\
Since $H^0(\mathcal{E}(-1))=H^1(\mathcal{E}(-2))=0$, the long sequence yields
\begin{equation}
0=H^0(\mathcal{E}(-1)) \to H^0(\mathcal{E}(-1)_{|p}) \to H^1(\mathcal{E}(-2))=0
\end{equation}
So $H^0(\mathcal{E}(-1)|_H)=0$ and therefore $H^0(\mathcal{E}(k)|_H)=0$ for $k \leq -1$ . Then for $k \leq -1$
\begin{equation}
0=H^0(\mathcal{E}(k)|_H) \to H^1(\mathcal{E}(k-1)) \to H^1(\mathcal{E}(k))
\end{equation}
By induction $H^1(\mathcal{E}(k))=0$ for $k \leq -2$. \\
By hypothesis $H^n(\mathcal{E}(-n))=H^{n-1}(\mathcal{E}(1-n))=0$ then $H^{n-1}(\mathcal{E}(1-n)|_H)=0$. \\
We apply the Beilinson spectral sequence on $\mathcal{E}(-1)$ to obtain a spectral sequence with
$\mathcal{E}^{p,q}_1=H^q(\mathcal{E} \otimes \Omega^{-p}_{\mathbb{P}^n}(-p) \otimes \mathcal{O}_{\mathbb{P}^n}(p) )$ \\
Now, suppose we had 
\begin{equation}\label{suporte}
H^q(\mathcal{E}(-1) \otimes \Omega^{-p}_{\mathbb{P}^n}(-p))=0 \ for \ q \ \not= 1 \ and \ for \ q=1, \ p\leq -3. 
\end{equation}
Lemma (\ref{lemabeilinson}) gives a monad
\begin{multline*}
0 \to H^1(\mathcal{E}(-1) \otimes \Omega^{2}_{\mathbb{P}^n}(2)) \otimes \mathcal{O}_{\mathbb{P}^n}(-2) \\ \to H^1(\mathcal{E}(-1) \otimes \Omega^{1}_{\mathbb{P}^n}(1)) \otimes \mathcal{O}_{\mathbb{P}^n}(-1) \to H^1(\mathcal{E}(-1)) \otimes \mathcal{O}_{\mathbb{P}^n} \to 0
\end{multline*}
With $\mathcal{E}(-1)$ as cohomology. So twisting it by $\mathcal{O}_{\mathbb{P}^n}(1)$ gives a monad with cohomology $\mathcal{E}$.
The only thing left to do is to prove (\ref{suporte}).
Consider the Euler sequence for p-forms 
\begin{equation}
0 \to  \Omega^{-p}_{\mathbb{P}^n}(-p) \to \mathcal{O}_{\mathbb{P}^n} ^{\oplus m} \to  \Omega^{-p-1}_{\mathbb{P}^n}(-p) \to 0
\end{equation}
where  $p=-1,...,-n$ and $m= {n+1\choose -p}$. and twist it by $\mathcal{E}(k)$
\begin{equation}
0 \to \mathcal{E}(k) \otimes \Omega^{-p}_{\mathbb{P}^n}(-p) \to \mathcal{E}(k) ^{\oplus m} \to \mathcal{E}(k) \otimes  \Omega^{-p-1}_{\mathbb{P}^n}(-p) \to 0
\end{equation}
We have the following vanishings \\
- $H^0( \mathcal{E}(k) \otimes \Omega^{-p}_{\mathbb{P}^n}(-p) )=0$ for every p and $k \leq -1$ \\
- $H^q(\mathcal{E}(-1) \otimes \Omega^{n}_{\mathbb{P}^n}(n))=H^q(\mathcal{E}(-2))=0$ for every q\\
- $H^q(\mathcal{E}(-1))=0$ for every $q\not=1$\\
$H^n(\mathcal{E}(k) \otimes \Omega^{-p}_{\mathbb{P}^n}(-p)) =0$ for every p and $k \geq -n$. Indeed, by hypothesis $H^0(\mathcal{E}(-1))=0$, hence $H^0( \mathcal{E}(k) \otimes \Omega^{-p}_{\mathbb{P}^n}(-p) )=0$ for every p and $k \leq -1$. The second equation follows from the fact that $\Omega^{n}_{\mathbb{P}^n}(n)\cong\mathcal{O}_{\mathbb{P}^n}(-1)$. 
\end{proof}
%\begin{definition}A linear sheaf $\mathcal{E}$ with $c_1(\mathcal{E})=0$ is an \textbf{instanton sheaf}. \\
%The integer $c=- \chi(\mathcal{E}(-1))$ is called the charge of $\mathcal{E}$
%\end{definition}
Now that we have established a characterization of a linear monad, it is natural to ask whether we can guarantee any further property. We already know $\mathcal{E}$ is locally free precisely when its degeneration locus is empty. 
We will prove that the degeneration locus of a monad is a subvariety. %from there 
%For instance, we could ask when $E$ is just reflexive, or just torsion free.
%Next lemma tell us that we can look at the codimension of the degeneration locus.

\begin{definition}\label{homologicaldimension}
Let $M$ be a $A$-module where $A$ is a ring, if $M$ is finitely generated then it has a projective resolution $0 \to P_n \to ... \to P_0 \to M \to 0$ with $P_n \not= 0$. The homological dimension (or projective dimension) of $M$ is defined to be the smallest number $n$. We can define a similar concept for coherent sheaves:
let $\mathcal{F}$ be a coherent sheaf over $X=\mathbb{P}^n$ and $x \in X$. The stalk $\mathcal{F}_x$ is a finitely generated module over the local noetherian ring $\mathcal{O}_{X,x}$. Hence we can define the \textbf{homological dimension} $dh(\mathcal{F}_x)$ over $\mathcal{O}_{X,x}$ to be the minimal length of a projective (thus free) resolution of $\mathcal{F}_x$. \\
Let $M$ be a module over a semi-local ring $A$ and $\mathfrak{J}$ the Jacobson radical of $A$. An $M$-sequence is a sequence $(a_1,...,a_p)$, $a_i \in \mathfrak{J}$ satisfying\\
- For every integer $i$ with $1 \leq i \leq p$ we have: $a_i$ is not a zero divisor in $M/(a_1,...,a_{i-1})M$.\\
The cohomological dimension $codh(\mathcal{F}_x)$ is defined as the maximal length of an $\mathcal{F}_x$-sequence in $\mathcal{O}_{X,x}$.The detailed discussion can be found in \cite[Serre, IV-14]{serre_1975}.
\end{definition}
The integer $dh(\mathcal{F}_x)$ is the smallest number $k$ so that for all finitely generated $\mathcal{O}_{X,x}$-module $M$ and all $i>k$ we have $Ext^i_{\mathcal{O}_{X,x}}(\mathcal{F}_x,M)=0$. We briefly explain it. \\
Let $0 \to P_n \overset{d_n}{\to} P_{n-1} \to ... \to P_1 \to P_0 \overset{d_0}{\to} \mathcal{F}_x$ be a minimal free resolution of $\mathcal{F}_x$. Let $M$ be any $\mathcal{O}_{X,x}$-module (it is sufficient to consider $M$ finitely generated). Consider the exact sequences $0 \to K_i \to P_i \to K_{i-1} \to 0$ where $K_i = ker d_i=im d_{i+1}$ and apply the functor $Hom_{\mathcal{O}_{X,x}}(-,M)$. Any $Ext_{\mathcal{O}_{X,x}}^i(P_j,M)$ vanishes for $i>0$, thus setting $i=n-1$ we obtain $Ext_{\mathcal{O}_{X,x}}^1(K_{n-1},M) \cong Ext_{\mathbb{O}_{X,x}}^2(K_{n-2},M)=0$ because $K_{n-1} \cong P_n$. This process implies $Ext_{\mathcal{O}_{X,x}}^i(K_{n-i},M)\cong Ext_{\mathcal{O}_{X,x}}^{i+1}(K_{n-i-1},M)$ until the last sequence $0 \to K_0 \to P_0 \to \mathcal{F}_x$. Hence $Ext_{\mathcal{O}_{X,x}}^{n+1}(\mathcal{F}_x,M)=0$, while $Ext_{\mathcal{O}_{X,x}}^n(\mathcal{F}_x,M) \not = 0$ comes from the minimality of the resolution.
\begin{lemma}\cite[Okonek, Schneider \& Spindler, Lemma 1.1.1]{Okonek}\label{dhokonek}
$dh(\mathcal{F}_x) \leq q$ if and only if for all $i>q$ we have $(\mathcal{E}xt^i(\mathcal{F},\mathcal{O}_X))_x=0$ for all $x \in X$
\end{lemma}
\begin{lemma}\cite[Jardim]{jardim}
Let $\mathcal{E}$ be a linear sheaf, then \\
(i) $\mathcal{E}$ is locally free if and only if its degeneration locus is empty \\
(ii) $\mathcal{E}$ is reflexive if and only if its degeneration locus is a subvariety of codimension at least 3 \\
(iii) $\mathcal{E}$ is torsion free if and only if its degeneration locus is a subvariety of codimension at least 2
\end{lemma}
\begin{proof}
Let $\Sigma$ be its degeneration locus.
We have already proved (i). \\
To prove (ii) and (iii) we use the concept of the $m$-th singularity set of a coherent sheaf $\mathcal{F}$ : \\
$S_m (\mathcal{F}) = \{ x \in \mathbb{P}^n | codh \ \mathcal{F}_x \leq m \} = \{ x \in \mathbb{P}^n | dh(\mathcal{F}_x) \geq n-m \} $ where $dh(\mathcal{F}_x)$ denotes the homological dimension of $\mathcal{F}_x$ as an $\mathcal{O}_x$-module. \cite[Okonek, Schneider \& Spindler, Lemma 1.1.3]{Okonek}. Lemma (\ref{dhokonek})  tells us that
\begin{equation}
dh(\mathcal{F}_x)=d \iff \begin{cases} Ext^d (\mathcal{F}_x, \mathcal{O}_{X,x}) \not= 0 \\ Ext^p(\mathcal{F}_x,\mathcal{O}_{X,x})=0 \ \forall p >d \end{cases} 
\end{equation}
Back to our case, $x \in \Sigma$ implies $dh(\mathcal{E}_x)=1$, and $x \not \in \Sigma$ implies $dh(\mathcal{E}_x)=0$. Then $S_0(\mathcal{E}) = ... = S_{n-2}(\mathcal{E}) = \emptyset$ and $S_{n-1}=\Sigma$.\\
%\textcolor{red}{não achei a referencia dos dois itens abaixo}\\
Recall that a coherent sheaf $\mathcal{F}$ is a $k$-th syzygy sheaf if there is an exact sequence
\begin{equation}
0 \to \mathcal{F} \to \mathcal{O}_X^{p_1} \to \mathcal{O}_X^{p_2} \to ... \to \mathcal{O}_X^{p_k}
\end{equation}
\cite[Siu, Proposition 1.20]{Siu} states that:\\
(i) if $codim \ \Sigma \geq 2$ then $dim \ S_m(\mathcal{E}) \leq m-1 \ \forall m<n$, hence $\mathcal{E}$ is a locally $1^{st}$-syzygy sheaf \\
(ii) if $codim \ \Sigma \geq3$ then $dim \ S_m(\mathcal{E}) \leq m-2 \ \forall m<n$, hence $\mathcal{E}$ is a locally $2^{nd}$-syzygy sheaf. \\
Finally we use that $\mathcal{E}$ is torsion free if and only if it is a locally $1^{st}$-syzygy sheaf, and $\mathcal{E}$ is reflexive if and only if it is a locally $2^{nd}$-syzygy sheaf. \cite[Okonek, Schneider \& Spindler, p.148-149]{Okonek}
\end{proof}

\section{Horrocks monads}
In this section $X$ is a projective variety over $k$ with a given very ample invertible sheaf $\mathcal{O}_X(1)$. From this section and until the end of the dissertation a monad will \textbf{always have} cohomology $\mathcal{E}$ with $Sing(\mathcal{E})=\emptyset$ (thus it is always locally free). \\
It is interesting to notice that if $\mathcal{E}$ is locally free then $\alpha^*$ is surjective, so a monad 
\begin{equation}
\mathcal{V}_\bullet : \mathcal{V}_0 \overset{ \alpha }{ \to} \mathcal{V}_1 \overset{ \beta }{ \to} \mathcal{V}_2
\end{equation}
gives another monad by dualizing $V_\bullet$ :
\begin{equation}
\mathcal{V}_\bullet ^* : \mathcal{V}_2 ^* \overset{ \beta ^* }{ \to} \mathcal{V}_1 ^* \overset{ \alpha ^* }{ \to} \mathcal{V}_0 ^*
\end{equation}
Denote $H^1_*(\mathcal{F})=\bigoplus_{k\in \mathbb{Z}}H^1(\mathcal{F}(k))$. We will now define a Horrocks monad as follows
\begin{definition}\label{horrocksdef1}\cite[Jardim \& Martins]{jardim2}
A monad is said to be Horrocks if \\
(i) $\mathcal{V}_0 = \bigoplus\limits_{i=1}^r \omega_X (k_i)$ for $k_i \in \mathbb{Z}$ \\
(ii) $\mathcal{V}_2 = \bigoplus\limits_{j=1}^s \mathcal{O}_X (l_j)$ for $l_j \in \mathbb{Z}$ \\
(iii) $H^1_{*}(\mathcal{V}_1)=H^{n-1}_{*}(\mathcal{V}_1)=0$ \\
\end{definition}
We will now define an ACM variety
\begin{definition}\label{ACM}\cite[Eisenbud]{eisenbud}
A projective variety $X \xhookrightarrow{} \mathbb{P}^r$ of pure dimension $n$ is arithmetically Cohen-Macaulay (ACM) if its homogeneous coordinate ring $S(X)=H^0_{*}(\mathcal{O}_X)$ is a Cohen-Macaulay ring
\end{definition}
This is equivalent to saying that $H^1_{*}(\mathbb{P}^r,\mathcal{I}_X)=0$ and $H^p_{*} (\mathcal{O}_X)=0$ for $1 \leq p \leq n-1$ where $\mathcal{I}_X$ is the saturated ideal sheaf of $X$.

%We need to prove one lemma before we are able to prove the next theorem
%\begin{lemma}\cite[Serre, p.2-08]{serre_2003}
%	Let $M$ be a $A$-module with $A$ a ring. Suppose $M$ has homological dimension $dh(M) \leq 1$ and let $\xi \in Ext^1_A(M,A)$. Let $E_\xi$ be the corresponding extension of $M$ given by the exact sequence $0 \to A \to E_\xi \to M \to 0$. Then $E_\xi$ is projective if and only if $\xi$ generates $Ext^1_A(M,A)$.
%\end{lemma}
%\begin{proof}
%	Notice that $dh(E_\xi) \leq 1$, by (\ref{homologicaldimension}) we have $Ext^1_A(E_\xi,A)=0$ if and only if $E_\xi$ is projective. Applying the functor $Hom_A(-,A)$ we have $Hom_A(A,A) \to Ext^1_A(M,A) \to Ext^1_A(E_\xi,A) \to 0$. In other words $A \overset{.\xi}{\to} Ext^1_A(M,A) \to Ext^1_A(E_\xi,A) \to 0$ where the map $.\xi$ denotes the multiplication $x \mapsto x\xi$, and such map is surjective if and only if $\xi$ generates $Ext^1_A(M,A)$.
%\end{proof}
\begin{theorem}\cite[Jardim \& Martins, Theorem 2.3]{jardim2}\label{Horrocks1}
Let $X$ be an ACM variety of dimension $n \geq 3$ and let $\mathcal{E}$ be a locally free sheaf on $X$. There exists a bijective correspondence between collections {$h_1,...,h_r,g_1,...,g_s$} with $h_i \in H^1(\mathcal{E}^* \otimes \omega_X (k_i))$ and $g_j \in H^1(\mathcal{E}(-l_j))$, for integers $k_i$ and $l_j$, and isomorphism classes of monads of the form
\begin{equation}\label{eqhorrocks}
\mathcal{V}_\bullet : \bigoplus\limits_{i=1}^r \omega_X (k_i) \overset{ \alpha }{ \to} \mathcal{F} \overset{ \beta }{ \to} \bigoplus\limits_{j=1}^s \mathcal{O}_X (l_j)
\end{equation}
whose cohomology is isomorphic to $\mathcal{E}$. This correspondence is such that $\mathcal{V}_\bullet$ is Horrocks if and only if the $g_j$ generate $H^1_{*}(\mathcal{E})$ and the $h_i$ generate $H^1_{*}(\mathcal{E}^* \otimes \omega_X)$ as $S(X)$-modules.
\end{theorem}
\begin{proof}
Let $\Sigma_i h_i \in \bigoplus_i H^1 (\mathcal{E}^* \otimes \omega_X (k_i)) \cong H^1(\mathcal{E}^* \otimes (\bigoplus_i \omega_X (k_i)) \cong \\ Ext^1 (\mathcal{O}_X,\mathcal{E}^* \otimes \bigoplus_i \omega_X (k_i)) \cong Ext^1(\mathcal{E},\bigoplus_i \omega_X (h_i))$. This element defines an extension 
\begin{equation}\label{eqhorrocks1}
0 \to \bigoplus_{i=1}^r \omega_X (k_i) \overset{\alpha}{\to} \mathcal{K} \to \mathcal{E} \to 0
\end{equation}
In particular we have, for every $p$, the following exact sequence in cohomology
\begin{equation}
\bigoplus_{i=1}^r H^p( \omega_X (k_i)) \to H^p(\mathcal{K}) \to H^p(\mathcal{E}) \to \bigoplus_{i=1}^r H^{p+1}( \omega_X (k_i))
\end{equation}
But $0\overset{ACM}{=}H^{n-p}(\mathcal{O}_X (-k_i)) \cong H^p(\omega_X(k_i))$ hence $H^p (\mathcal{K}(m)) \cong H^p (\mathcal{E}(m))$ \\ for $1 \leq p \leq n-2$ and any $m$. \\
Let $g_j ' \in H^1 (\mathcal{K}(-l_j))$ be the image through the isomorphism above for every $g_j$. We have the isomorphisms $ \bigoplus_j H^1(\mathcal{K}(-l_j)) \cong H^1(\bigoplus_j \mathcal{K}(-l_j)) \cong Ext^1 (\mathcal{O}_X,\bigoplus_j \mathcal{K}(-l_j) ) \cong Ext^1(\bigoplus_j \mathcal{O}_X (l_j),\mathcal{K})$, so $\Sigma_j g_j' \in \bigoplus_j H^1(\mathcal{K}(-l_j))$ defines an extension :
\begin{equation}\label{eqhorrocks2}
0 \to \mathcal{K} \to \mathcal{F} \overset{\beta}{\to} \bigoplus_{j=1}^{s} \mathcal{O}_X (l_j) \to 0
\end{equation}
Putting (\ref{eqhorrocks1}) and (\ref{eqhorrocks2}) together gives the monad :
\begin{equation}\label{eqhorrocks3}
\bigoplus_{i=1}^r \omega_X (k_i) \overset{\alpha}{\to} \mathcal{F} \overset{\beta}{\to} \bigoplus_{j=1}^{s} \mathcal{O}_X (l_j)
\end{equation}
Since $\mathcal{K}=ker \ \beta$ and $coker \ \alpha = \mathcal{E}$ we have $\mathcal{E}$ as the cohomology of (\ref{eqhorrocks3}).
Now, conversely if we have an isomorphism class as in (\ref{eqhorrocks}), then there are two exact sequences 
\begin{equation}
0 \to \bigoplus\limits_{i=1}^r \omega_X (k_i) \to \mathcal{K} \to \mathcal{E} \to 0
\end{equation}
and
\begin{equation}\label{auxiliar}
0 \to \mathcal{K} \to \mathcal{F} \to \bigoplus\limits_{j=1}^s \mathcal{O}_X (l_j) \to 0
\end{equation}
with $\mathcal{K} = ker \beta$. The first one corresponds to an element $h \in Ext^1(\mathcal{E}, \bigoplus_i \omega_X (k_i) ) \cong Ext^1 (\mathcal{O}_X,\mathcal{E}^* \otimes \bigoplus_i \omega_X (k_i)) \cong H^1(\mathcal{E}^* \otimes \bigoplus_i \omega_X (k_i) )$, while the second sequence corresponds to an element $g \in Ext^1(\bigoplus\limits_{j=1}^s \mathcal{O}_X (l_j),\mathcal{K}) \cong \bigoplus_j H^1( \mathcal{K}(-l_j)) \cong \bigoplus_j H^1(\mathcal{E}(-l_j))$. Set $h = \Sigma_i h_i$ and $g = \Sigma_j g_j$, we still need to prove the last statement. \\
First notice that conditions (i) and (ii) of (\ref{horrocksdef1}) are satisfied automatically.
Since $X$ is ACM, $H^{1}_* (\bigoplus_j \mathcal{O}_X (l_j))=0$. But then using the exact sequence (\ref{auxiliar}) we have the following exact sequence :
\begin{equation}
H^0(\bigoplus_j \mathcal{O}_X (l_j +k)) \overset{\delta_k}{\to} H^1(\mathcal{K}(k)) \twoheadrightarrow H^1 (\mathcal{F}(k)) \to 0
\end{equation} 
Hence $\delta_k$ is surjective for every $k$ if and only if $H^{1}_* (\mathcal{F})=0$.
Those morphisms are defined by construction taking $\Sigma f_j$ to $\Sigma  f_j g_j'$, each $f_j\in \mathcal{O}_X (l_j +k)$ has degree $l_j+k$, this is because given an element $\xi \in Ext^1(\mathcal{O}_X,\mathcal{K})$ corresponding to an extension 
\begin{equation}
0 \to \mathcal{K} \to \mathcal{F} \to \mathcal{O}_X \to 0
\end{equation}
And an element $g \in H^1(\mathcal{K}) \cong Ext^1(\mathcal{O}_X,\mathcal{K})$. 
Then the morphism $H^0(\mathcal{O}_X) \to H^1(\mathcal{K})$ has to send $1$ to $g$.\\
Surjectivity of each $\delta_k$ is equivalent to saying that $g_j'$ generates $H^1_*(\mathcal{K})$ as $H^0_*(\mathcal{O}_X)$-module. \\
Finally, using Serre duality $H^{n-1}(\mathcal{F}(k)) \cong Ext^1(\mathcal{F}(k),\omega_X)^* \cong Ext^1(\mathcal{O}_X,\mathcal{F}^* \otimes \omega_X (-k))^* \cong H^1(\mathcal{F}^* \otimes \omega_X (-k))^*$ hence $H^{n-1}_*(\mathcal{F})=0$ if and only if $H^1_*(\mathcal{F}^* \otimes \omega_X)=0$.
Since $\mathcal{E}$ is locally free the dual of $\mathcal{V}_\bullet$ is a monad, tensoring it by $\omega_X$ gives
\begin{equation}
\mathcal{V}_\bullet ^* :\bigoplus\limits_{j=1}^s \omega_X (-l_j) \to \mathcal{F}^* \otimes \omega_X \to \bigoplus\limits_{i=1}^r \mathcal{O}_X (-k_i)
\end{equation}
Which gives
\begin{equation}
H^0(\bigoplus\limits_{i=1}^r \mathcal{O}_X (-k_i+k)) \overset{\delta_k}{\to} H^1(\bigoplus\limits_{j=1}^s \omega_X (-l_j+k)) \to H^1(\mathcal{F}^* \otimes \omega_X(k)) \to 0
\end{equation}
The same argument shows that $H^{n-1}_* (\mathcal{F})=0$ if and only if $h_i$ generate $H^1 _* (\mathcal{E}^* \otimes \omega_X)$ as $H^0 _* (\mathcal{O}_X)$-module.
\end{proof}
The theorem leads to an interesting result not only concerning the case $X=\mathbb{P}^n$, but this time considering $X$ as an ACM variety of dimension $\geq 3$.
\begin{corollary}\cite[Jardim \& Martins]{jardim2}
Every locally free sheaf $\mathcal{E}$ on an ACM variety of dimension $n \geq 3$ is the cohomology of a monad
\begin{equation}
\mathcal{V}_\bullet : \bigoplus\limits_{i=1}^r \omega_X (k_i) \overset{ \alpha }{ \to} \mathcal{F} \overset{ \beta }{ \to} \bigoplus\limits_{j=1}^s \mathcal{O}_X (l_j)
\end{equation}
such that \\
(i) $H^1_*(\mathcal{F}) = H^{n-1}_*(\mathcal{F})=0$ \\
(ii) for $n \geq 4$, $H^p_* (\mathcal{F}(k)) \cong H^p_*(\mathcal{E}(k))$ for $2 \leq p \leq n-2$
\end{corollary}
\begin{proof}
$\mathcal{E}$ is locally free so $H^1_*(\mathcal{E})$ and $H^1_*(\mathcal{E}^* \otimes \omega_X)$ are finitely generated as $H^0_*(\mathcal{O}_X)$-modules. Then we can pick any set of generators and apply last theorem to generate the desired monad.
\end{proof}
Since every Horrocks monad is a complex, and every morphism of Horrocks monads is a morphism of complexes it is immediate to see that Horrocks monads on $X$ form a full subcategory $\mathcal{H}(X)$ of $Ch_\bullet(X)$ the abelian category of complexes of sheaves on $X$. Since $\mathcal{H}(X)$ is a full subcategory of an abelian category, we also have that $\mathcal{H}(X)$ is additive.



\begin{remark}
Let $\mathcal{V}_\bullet$ and $\mathcal{W}_\bullet$ be two monads and $\phi_\bullet : \mathcal{V}_\bullet \to \mathcal{W}_\bullet$ be a morphism of monads. Then $\phi_\bullet$ defines a morphism of sheaves between the cohomologies of $\mathcal{V}_\bullet$ and $\mathcal{W}_\bullet$. \\
Moreover, this defines a functor $\textbf{C}:\mathcal{H}(X) \to \mathcal{V}(X)$ from the category of Horrocks monads to the category of locally free sheaves on $X$.
\end{remark}
\begin{proof}
Let $\mathcal{E}$ and $\mathcal{F}$ be the cohomology of $V_\bullet$ and $W_\bullet$ respectively. Considering both monads as complexes over $\mathcal{V}(X)$ we obtain a morphism which we call $\textbf{C}(\phi):\mathcal{E} \to \mathcal{F}$. Any composition of morphisms of complexes $\phi_\bullet \circ \psi_\bullet$ goes to $\textbf{C}(\phi_\bullet \circ \psi_\bullet)=\textbf{C}(\phi)\circ \textbf{C}(\psi)$ since everything is done in $Ch_\bullet(X)$.
\end{proof}
Theorem (\ref{Horrocks1}) states that $\textbf{C}$ is surjective on objects, but more can be asked of it.
\begin{theorem}\cite[Jardim \& Martins, Theorem 2.5]{jardim2}\label{Horrocks2}
Let $X$ be an ACM projective variety and
\begin{align}
\mathcal{M}_\bullet : \mathcal{M}_0 \overset{\alpha}{\to} \mathcal{M}_1 \overset{\beta}{\to} \mathcal{M}_2  \\
\mathcal{N}_\bullet : \mathcal{N}_0 \overset{\alpha'}{\to} \mathcal{N}_1 \overset{\beta'}{\to} \mathcal{N}_2
\end{align}
be Horrocks monads. Consider the morphism
\begin{align*}
\rho : Hom(\mathcal{M}_1,\mathcal{N}_0) \oplus Hom(\mathcal{M}_2,\mathcal{N}_1) \to Hom(\mathcal{M}_\bullet,\mathcal{N}_\bullet) \\
(\psi_1,\psi_2) \mapsto (\psi_1 \alpha, \alpha' \psi_1 + \psi_2 \beta, \beta' \psi_2)
\end{align*}
Then the sequence
\begin{equation}
0 \to Im (\rho) \overset{i}{\to} Hom(\mathcal{M}_\bullet, \mathcal{N}_\bullet) \overset{\pi}{\to} Hom(\textbf{C}(\mathcal{M}_\bullet), \textbf{C}(\mathcal{N}_\bullet)) \to 0
\end{equation}
is exact, where $i$ is the inclusion and $\pi$ the natural morphism.
\end{theorem}
\begin{proof}

First we need to prove that $\rho$ is well defined. Denote $\gamma_0 =\psi_1 \alpha $, $\gamma_1 = \alpha' \psi_1 + \psi_2 \beta$ and $\gamma_2 = \beta' \psi_2$.
\begin{center}
\begin{tikzcd}
\mathcal{M}_0 \arrow[r, "\alpha"] \arrow[d, "\gamma_0"]
& \mathcal{M}_1 \arrow[r, "\beta"] \arrow[d, "\gamma_1"]
& \mathcal{M}_2 \arrow[d, "\gamma"] \\
\mathcal{N}_0 \arrow[r, "\alpha '"]
& \mathcal{N}_1 \arrow[r, "\beta '"]
& \mathcal{N}_2
\end{tikzcd}
\end{center}
But then $\gamma_1 \alpha = \alpha' \psi_1 \alpha + \psi_2 \beta \alpha \overset{\beta \alpha=0}{=} \alpha' \psi_1 \alpha = \alpha ' \gamma_0$ and $\beta' \gamma_1 = \beta' \alpha' \psi_1 + \beta' \psi_2 \beta \overset{\beta' \alpha'=0}{=} \gamma_2 \beta$, hence $\rho$ is well defined. \\
The map $i$ is injective by definition of image. To show $\pi i =0$ we can apply diagram chasing because the category of sheaves on an abelian category is an abelian category: Let $\mathcal{E}$ and $\mathcal{F}$ be the cohomology of $\mathcal{M}_\bullet$ and $\mathcal{N}_\bullet$ respectively, let $x + Im \alpha \in \mathcal{E}=ker \beta / Im \alpha$. \\ If $\gamma=(\gamma_0,\gamma_1,\gamma_2)=\rho(\psi_1,\psi_2)$ then $H(\gamma)(x+Im \alpha)=\gamma_1 (x) + Im \alpha' = \alpha' \psi_1 (x) + \psi_2 \beta (x) + Im \alpha'=0$ because $\alpha' (\psi_1(x)) \in Im\alpha'$ and $x \in ker \beta$ hence $\beta(x)=0$. \\
To establish the surjectivity of $\pi$, consider the exact sequences :
\begin{align*}
(i) \ 0 \to ker \beta \overset{j}{\to} \mathcal{M}_1 \overset{\beta}{\to} \mathcal{M}_2 \to 0 \\
(ii) \ 0 \to ker \beta' \overset{j'}{\to} \mathcal{N}_1 \overset{\beta'}{\to} \mathcal{N}_2 \to 0 \\
(iii) \ 0 \to \mathcal{M}_0 \overset{\alpha}{\to} ker \beta \overset{p}{\to} \mathcal{E} \to 0 \\
(iv) \ 0 \to \mathcal{N}_0 \overset{\alpha'}{\to} ker \beta' \overset{p'}{\to} \mathcal{F} \to 0
\end{align*}
Let $\phi : \mathcal{E} \to \mathcal{F}$, define $\tilde{\phi} =\phi \circ p : ker \beta \to \mathcal{F}$. Applying the functor $Hom (ker \beta, -)$ to the sequence (iv) results
\begin{equation}
Hom(ker \beta, ker \beta') \to Hom (ker \beta , \mathcal{F}) \to Ext^1(ker \beta, \mathcal{N}_0)
\end{equation}
Now, $Ext^1(ker \beta, \mathcal{N}_0)=0$, indeed applying $Hom(-,\mathcal{N}_0)$ to the sequence (i) we get
\begin{equation}
Ext^1(\mathcal{M}_1,\mathcal{N}_0) \to Ext^1(ker \beta, \mathcal{N}_0) \to Ext^2(\mathcal{M}_2, \mathcal{N}_0) 
\end{equation} But $Ext^2(\mathcal{M}_2,\mathcal{N}_0) \cong Ext^2(\mathcal{O}_X,\mathcal{M}_2^* \otimes \mathcal{N}_0)\cong H^2(\mathcal{M}_2^* \otimes \mathcal{N}_0)=0$ because $\mathcal{M}_2$ is a sum of line bundles, $\mathcal{N}_0$ is a sum of twists of the canonical bundle, so we would have a sum of terms $H^2(\omega_X(k_j))$ which are zero since $X$ is ACM. The same goes for $Ext^1(\mathcal{M}_1,\mathcal{N}_0)\cong H^1(\mathcal{M}_1^* \otimes \mathcal{N}_0)=0$ by definition of a Horrocks monad. But then $Ext^1(ker \beta, \mathcal{N}_0)=0$ and hence $Hom(ker \beta, ker \beta') \to Hom(ker \beta,\mathcal{F})$ is surjective so there is a map $\Phi \in Hom(ker \beta, ker \beta')$ with $\tilde{\phi}=p' \circ \Phi$, in other words the right square in the diagram below is commutative :
\begin{center}
\begin{tikzcd}
0 \arrow[r]
& \mathcal{M}_0 \arrow[r, "\alpha"]
& ker \beta \arrow[r] \arrow[d, "\Phi"] \arrow[rd, dashrightarrow , "\tilde{\phi}"]
& \mathcal{E} \arrow[r] \arrow[d, "\phi"]
& 0 \\
0 \arrow[r]
& \mathcal{N}_0 \arrow[r, "\alpha '"]
& ker \beta ' \arrow[r]
& \mathcal{F} \arrow[r]
& 0
\end{tikzcd}
\end{center}
Applying $Hom(\mathcal{M}_0,-)$ to the sequence (iv) results in 
\begin{equation}
0 \to Hom(\mathcal{M}_0,\mathcal{N}_0) \to Hom(\mathcal{M}_0, ker \beta') \to Hom(\mathcal{M}_0,\mathcal{F}) \to Ext^1(\mathcal{M}_0,\mathcal{N}_0) = 0
\end{equation}
exact since since $X$ is $ACM$. But $p' \circ \Phi \circ \alpha =\tilde{\phi} \circ \alpha = \phi \circ p \circ \alpha=0$ because $p \circ \alpha =0$ so $\Phi \circ \alpha$ is in the kernel of the second morphism. Therefore there exists a map $g \in Hom(\mathcal{M}_0,\mathcal{N}_0)$ such that $\alpha ' g = \Phi \alpha$ hence making the following diagram commutative.
\begin{center}
\begin{tikzcd}
0 \arrow[r]
& \mathcal{M}_0 \arrow[r, "\alpha"] \arrow[d,"g"]
& ker \beta \arrow[r] \arrow[d, "\Phi"] \arrow[rd, dashrightarrow , "\tilde{\phi}"]
& \mathcal{E} \arrow[r] \arrow[d, "\phi"]
& 0 \\
0 \arrow[r]
& \mathcal{N}_0 \arrow[r, "\alpha '"]
& ker \beta ' \arrow[r]
& \mathcal{F} \arrow[r]
& 0
\end{tikzcd}
\end{center}
The composition of $\Phi$ with the first map in (ii) gives a map $\tilde{\Phi}:ker \beta \to \mathcal{N}_1$. Applying the functor $Hom(-,\mathcal{N}_1)$ to the sequence (i) yields
\begin{equation}
Hom(\mathcal{M}_1,\mathcal{N}_1) \to Hom(ker \beta, \mathcal{N}_1) \to Ext^1(\mathcal{M}_2,\mathcal{N}_1)=0
\end{equation}
So there is a $h : \mathcal{M}_1 \to \mathcal{N}_1$ such that the left square in the diagram below commutes :
\begin{center}
\begin{tikzcd}
0 \arrow[r]
& ker \beta \arrow[r] \arrow[d,"\Phi"] \arrow[rd, dashrightarrow, "\tilde{\Phi}"]
& \mathcal{M}_1 \arrow[r, "\beta"] \arrow[d, "h"] 
& \mathcal{M}_2 \arrow[r]
& 0 \\
0 \arrow[r]
& ker \beta' \arrow[r]
& \mathcal{N}_1 ' \arrow[r, "\beta '"]
& \mathcal{N}_2 \arrow[r]
& 0
\end{tikzcd}
\end{center}
Applying the functor $Hom(-,\mathcal{N}_2)$ to (i)
\begin{equation}
0 \to Hom(\mathcal{M}_2, \mathcal{N}_2) \to Hom(\mathcal{M}_1,\mathcal{N}_2) \to Hom(ker \beta, \mathcal{N}_2) \to Ext^1(\mathcal{M}_2, \mathcal{N}_2) = 0
\end{equation}
Using the same argument as before we have that $\beta ' \circ h$ is in the kernel of the second map, so there is a map $l : \mathcal{M}_2 \to \mathcal{N}_2$ with $\beta ' \circ h = l \circ \beta$. Thus making the diagram commute
\begin{center}
\begin{tikzcd}
0 \arrow[r]
& ker \beta \arrow[r] \arrow[d,"\Phi"] \arrow[rd, dashrightarrow, "\tilde{\Phi}"]
& \mathcal{M}_1 \arrow[r, "\beta"] \arrow[d, "h"] 
& \mathcal{M}_2 \arrow[r] \arrow[d,dashrightarrow, "l"]
& 0 \\
0 \arrow[r]
& ker \beta' \arrow[r]
& \mathcal{N}_1 ' \arrow[r, "\beta '"]
& \mathcal{N}_2 \arrow[r]
& 0
\end{tikzcd}
\end{center}
By construction $(g,h,l)$  is a morphism of complexes from $\mathcal{M}_\bullet$ to $\mathcal{N}_\bullet$.  \\ Notice that $\pi (g,h,l)=\phi$ :  if $x \in \mathcal{E}$ then $\pi(g,h,l)(x+Im \alpha)=h(x)+Im \alpha '=p' \circ h(x)=p' \circ h \circ j(x)=p' \circ \tilde{\Phi}(x)=p' \circ j' \circ \Phi (x) = p' \circ \Phi (x) = \phi \circ p(x) = \phi (x + Im \alpha)$,
in which we are considering $j(x)=x$ and $j'(y)=y$ since $j,j'$ are inclusions. \\
Therefore $\pi$ is surjective. To prove exactness suppose $(g,h,l) \in ker \pi$, then we get the following diagram :
\begin{center}
\begin{tikzcd}
0 \arrow[r]
& \mathcal{M}_0 \arrow[r, "\alpha"] \arrow[d,"g"]
& ker \beta \arrow[r, "p"] \arrow[d, "h'"]
& \mathcal{E} \arrow[r] \arrow[d, "0"]
& 0 \\
0 \arrow[r]
& \mathcal{N}_0 \arrow[r, "\alpha '"]
& ker \beta ' \arrow[r, "p'"]
& \mathcal{F} \arrow[r]
& 0
\end{tikzcd}
\end{center}
where we consider the restriction of $h$ as $h': ker \beta \to ker \beta '$.
Applying the functor $Hom(ker \beta , -)$ on the sequence (iv) yields
\begin{equation}
0 \to Hom(ker \beta, \mathcal{N}_0) \to Hom(ker \beta , ker \beta ') \to Hom(ker \beta , \mathcal{F}) \to Ext^1(ker \beta , \mathcal{N}_0)=0
\end{equation}
But the diagram above states that $p' \circ h'=0$ so $h' \in ker (Hom(ker \beta,\mathcal{N}_0) \to Hom(ker \beta , ker \beta '))$ hence there exists a map $\psi_1: ker \beta \to \mathcal{N}_0$ with $\alpha ' \psi_1 = h'$.
\begin{center}
\begin{tikzcd}
0 \arrow[r]
& \mathcal{M}_0 \arrow[r, "\alpha"] \arrow[d,"g"]
& ker \beta \arrow[r, "p"] \arrow[d, "h'"] \arrow[ld,dashrightarrow,"\psi_1"]
& \mathcal{E} \arrow[r] \arrow[d, "0"]
& 0 \\
0 \arrow[r]
& \mathcal{N}_0 \arrow[r, "\alpha '"]
& ker \beta ' \arrow[r, "p'"]
& \mathcal{F} \arrow[r]
& 0
\end{tikzcd}
\end{center}
Applying the functor $Hom(-,\mathcal{N}_0)$ to (i)
\begin{equation}
0 \to Hom(\mathcal{M}_2,\mathcal{N}_0) \to Hom(\mathcal{M}_1,\mathcal{N}_0) \to Hom(ker \beta, \mathcal{N}_0) \to Ext^1(\mathcal{M}_2,\mathcal{N}_0)=0
\end{equation}
Then there is a map $\Psi_1:\mathcal{M}_1 \to \mathcal{N}_0$ extending $\psi_1$. \\ Since $Im \alpha \subset ker \beta$, $\Psi \circ \alpha = \psi \circ \alpha$, by the last diagram $h' \circ \alpha=\alpha' \circ g$ hence $\alpha' \circ \psi_1 \circ \alpha=h' \circ \alpha = \alpha' \circ g$ which by injectivity implies $\psi_1 \circ \alpha = g$ hence $\Psi_1 \circ \alpha = g$ (so $g$ is the image of $\rho$ on the first coordinate). \\
Consider the map $H=h-\alpha'\circ\Psi_1:\mathcal{M}_1 \to \mathcal{N}_1$, the restriction of $H$ on $ker \beta$ is $h'-\alpha'\circ\psi_1=0$ so $H$ lies in the kernel of the second map in the exact sequence
\begin{equation}
Hom(\mathcal{M}_2,\mathcal{N}_1) \to Hom(\mathcal{M}_1,\mathcal{N}_1) \to Hom(ker \beta, \mathcal{N}_1)
\end{equation}
obtained from (i). But then there is $\Psi_2:\mathcal{M}_2 \to \mathcal{N}_1$ such that $H=\Psi_2 \circ \beta$ hence $h=\Psi_2 \beta + \alpha' \Psi_1$ (so $h$ is the image of $\rho$ on the second coordinate). \\
Finally, $l \beta = \beta' h = \beta'\Psi_2 \beta + \beta' \alpha' \Psi_1= \beta' \Psi_2 \beta$ where the first equality comes from the fact that $(g,h,l)$ is a morphism of complexes, and the third is due to $\beta' \alpha'=0$. But this implies $l=\beta' \Psi_2$ because $\beta$ is surjective (so $l$ is the image of $\rho$ on the third coordinate). \\
Those three equalities prove $(g,h,l)=(\Psi_1 \alpha, \alpha'\Psi_1 + \Psi_2 \beta, \beta' \Psi_2) \in Im \ \rho$ so the sequence is exact.
\end{proof}
\begin{theorem}\cite[Jardim \& Martins, Theorem 2.6]{jardim2}
Let $X$ be an ACM variety of dimension $n \geq 3$. The functor $\textbf{C}$ is additive, essentially surjective, full and exact.
\end{theorem}

\begin{proof}
It is essentialy surjective by Theorem (\ref{Horrocks1}), full by Theorem (\ref{Horrocks2}) and additive because $\pi(f+g)=\pi(f)+\pi(g)$ (the same $\pi$ in the last theorem). \\
Consider the following exact sequence of monads:
\begin{center}
\begin{tikzcd}
0 \arrow[d]
& 0 \arrow[d]
& 0 \arrow[d] \\
\mathcal{L}_0 \arrow[d,"g_1"] \arrow[r,"\alpha_L"]
& \mathcal{L}_1 \arrow[d,"h_1"] \arrow[r,"\beta_L"]
& \mathcal{L}_2 \arrow[d,"l_1"] \\
\mathcal{M}_0 \arrow[d,"g_2"] \arrow[r,"\alpha_M"]
& \mathcal{M}_1 \arrow[d,"h_2"] \arrow[r,"\beta_M"]
& \mathcal{M}_2 \arrow[d,"l_2"] \\
\mathcal{N}_0 \arrow[d] \arrow[r,"\alpha_N"]
& \mathcal{N}_1 \arrow[d] \arrow[r,"\beta_N"]
& \mathcal{N}_2 \arrow[d] \\
0
& 0
& 0
\end{tikzcd}
\end{center}
with $\mathcal{E}$, $\mathcal{F}$ and $\mathcal{G}$ the cohomology of the first, second and third rows. Let $\bar{h_1}:\mathcal{E} \to \mathcal{F}$ and $\bar{h_2}:\mathcal{F} \to \mathcal{G}$ be the induced maps. We have $\bar{h_2} \bar{h_1}=0$ because $\textbf{C}(\beta \alpha)=\textbf{C}(0)=0$. \\
Using the same argument from the proof of Theorem (\ref{Horrocks2}), we can verify the injectivity and surjectivity of $\bar{h_1}, \bar{h_2}$ respectively using diagram chasing : Let $\bar{x}=x+im \alpha_L \in \mathcal{E}$ and suppose $\bar{h_1}(\bar{x})=h_1(x)+im\alpha_M=0$. Then $h_1(x) \in im \alpha_M$ implies the existence of an unique $y \in \mathcal{M}_0$ with $\alpha_M(y)=x$ because $\alpha_M$ is injective. Using injectivity of $g_1$ gives another unique $z \in \mathcal{L}_0$ with $g_1(z)=y$. Since the diagram is commutative and $\alpha_L$ is injective we have $\alpha_L(z)=x$ hence $\bar{x}=0$. \\
Let $\bar{y}=y+im\alpha_N \in \mathcal{F}$.Being the map $h_2$ surjective, there is an element $x \in \mathcal{M}_1$ with $h_2(x)=y \in ker \beta_N$. Since every square is commutative, $l_2 \circ \beta_M(x)=0$ so $\beta_M(x) \in ker l_2$, but then $\beta_M(x) \in im l_1$ and thus there is an unique $z \in \mathcal{L}_2$ with $l_1(z)=\beta_M(x)$. Since $\beta_L$ is surjective there is a (non unique) $z' \in \mathcal{L}_1$ such that $\beta_L(z')=z$ which implies $\beta_M \circ h_1(z')=\beta_M(x)$.\\ Now the fact that $\alpha_L$ is a monomorphism implies there is an unique $z''$ with $\alpha_L(z'')=z'$ hence $\alpha_M \circ g_1(z'')=h_1(z')$. But then $h_1(z') \in ker \beta_M$ implies that $0 = l_1 \circ\beta_L(z')=l_1(z)$ which implies $0 = l_1(z)=\beta_M(x)$ hence $x \in ker \beta_M$ so the element $\bar{x}=x+im \alpha_M$ is well defined and $\bar{h_2}(\bar{x})=\bar{y}$. \\
Finally if $\bar{x} \in ker \bar{h_2}$ we have $h_2(x) \in im \alpha_N$ so there is an element $y \in \mathcal{N}_0$ with $\alpha_N(y)=h_2(x)$. But since $g_2$ is an epimorphism so there exists $ z \in \mathcal{M}_0$ such that $g_2(z)=y$ and $h_2 \circ \alpha_M(z)=\alpha_N \circ g_2(z)=\alpha_N(y)=h_2(x)$. So $h_2(\alpha_M(z)-x)=0$. \\ It is known \cite[MacLane, Theorem 3 (vi), p.201]{maclane} that implies $\alpha_M(z)-x \in im h_1$, which implies the existence of an element $z' \in \mathcal{L}_1$ with $h_1(z')=\alpha_M(z)-x$. Then $l_1 \circ \beta_L(z') =\beta_M(\alpha_M(z)-x)=0$ because $\beta_M \circ \alpha_M=0$ and $x \in ker \beta_M$. So $\beta_L(z') \in ker l_1$ implies $ \beta_L(z')=0$ hence $z' \in ker \beta_L$ and finally $\bar{z'}=z+im \alpha_L$ is well defined with $h_1(\bar{z'})=\bar{x}$ as desired.
\end{proof}
\begin{corollary}
If $X$ is an ACM variety of dimension $n \geq 3$ then the category $\mathcal{V}(X)$ of locally free sheaves on $X$ is equivalent to a quotient of the category $\mathcal{H}(X)$ of Horrocks monads on $X$.
\end{corollary}
With Theorem (\ref{Horrocks1}) in hands we are ready to begin the study of the classification of vector bundles.
\chapter{Toward the classification}
In this chapter we will introduce the spectrum of a reflexive sheaf and the Hartshorne-Serre correspondence. Then we will give a few families of curves which we can use the correspondence to obtain families of vector bundles of rank 2 with $c_1=-1$, each vector bundle has an spectrum and a monad associated.
%\chapter{Classification of stable rank 2 bundles with \texorpdfstring{$c_1=-1$}{texto}}
\section{Spectrum and its properties}
In \cite[Barth \& Elencwajg]{10.1007/BFb0063170} Barth and Elencwajg introduced the concept of the spectrum of a stable vector bundle $\mathcal{E}$ of rank 2 on $\mathbb{P}^3$ with $c_1=0$, we recall their construction. Let $L$ be a general line on $\mathbb{P}^3$, $p:X \to \mathbb{P}^3$ the blowing-up of $\mathbb{P}^3$ along $L$ and $q:X \to \mathbb{P}^1$ the morphism determined by the pencil of planes through $L$. The sheaf $\mathcal{H}:=R^1q_*p^*\mathcal{E}(-1)$ is locally free of rank $c_2$ on $\mathbb{P}^1$ \cite[Barth \& Elencwajg, Proposition 2.2.1 p.9]{10.1007/BFb0063170}. Grothendieck's theorem states that $\mathcal{H}$ is a finite sum of line bundles $\mathcal{O}_{\mathbb{P}^1}(k_1) \oplus ... \oplus \mathcal{O}_{\mathbb{P}^1}(k_{c_2})$ for suitable integers $k_i$. The sequence $\xi = \{k_i\}_{i=1,...,c_2}$ is called the \textbf{spectrum of $\mathcal{E}$}.

In \cite[Hartshorne, Theorem 7.1 p.151]{Hartshorne1980} Hartshorne generalized the concept of spectrum for any reflexive sheaf $\mathcal{F}$ of rank 2 with $c_1=0$ or $-1$ and $H^0(\mathcal{F}(-1))$ on $\mathbb{P}^3$.
\\In this chapter we won't be interested in the case $c_1=0$ since it has already been studied in \cite[Hartshorne \& Rao]{hartshorne1991}, but almost every result in this chapter is based on \cite[Hartshorne \& Rao]{hartshorne1991}. We start with some fundamental facts and definitions from Hartshorne's paper, starting from an equivalent definition for spectrum.
\begin{theorem}\cite[Hartshorne, Theorem 7.1]{Hartshorne1980}\label{alpinista1}
Let $\mathcal{F}$ be a rank 2 reflexive on $X$ with $c_1=0$ or $-1$. Assume $H^0(\mathcal{F}(-1))=0$. Then there is an unique sequence of integers (up to ordering) $\chi = \{k_i \}$, $i=1,...,c_2$ satisfying the following properties: \\
(1) $h^1(\mathbb{P}^3,\mathcal{F}(l))=h^0(\mathbb{P}^1,\mathcal{H}(l+1))$ for $l \leq -1$ \\
(2) $h^2(\mathbb{P}^3,\mathcal{F}(l))=h^1(\mathbb{P}^1,\mathcal{H}(l+1))$ for $l \geq -3$ if $c_1=0$ or $l \geq -2$ if $c_1=-1$ \\
Where $\mathcal{H}= \oplus \mathcal{O}_{\mathbb{P}^1}(k_i)$.
This set of integers is called the spectrum of $\mathcal{F}$.
\end{theorem}
In his proof Hartshorne gives a method on how to compute $\chi$.
Define  $n_l=h^1(\mathcal{F}(-l)) - h^1(\mathcal{F}(-l-1))$ for $l \geq 1$, then $n_l= \# \{ k_j \in \chi | k_j \geq l-1 \}$,
which results $n_l - n_{l-1}= \# \{k_j = l-1   \}$. In other words
\begin{equation}\label{spectrum}
\# \{k_j = l \}= h^1(\mathcal{F}(-l-1))-2h^1(\mathcal{F}(-l-2))+h^1(\mathcal{F}(-l-3)), \ l \geq 0
\end{equation}
We present some results which will help us in the computation of the spectrum
\begin{proposition}\cite[Hartshorne, Proposition 7.2]{Hartshorne1980}\label{alpinista2}
If $\mathcal{F}$ is locally free, then
\begin{align}
\{-k_i \}= \{k_i \} \ if \ c_1=0 \\
\{-k_i \} = \{k_i +1 \} \ if \ c_1=-1
\end{align}
\end{proposition}
\begin{theorem}\cite[Hartshorne, Theorem 7.5]{Hartshorne1980}\label{alpinista3}
Let $\mathcal{F}$ be as in the previous theorem and $\chi = \{k_j \}$ its spectrum. \\
(a) Assume $H^0 (\mathcal{F}(-1))=0$ \\
1) If there is a $k \geq 1$ in $\chi$, then 1,2,...,k also occur in $\chi$ \\
2) If there is a $k \leq -2$ then -1,-2,..,k also occur if $c_1=0$ and -2,-3,...,k also occur if $c_1=-1$ \\
(b) Assume $\mathcal{F}$ is stable. \\
1) If there is a $k \geq 1$ then 0,1,...,k also occur. \\
2) If there is a $k \leq -1$ then -1,-2,...,k also occur. Furthermore, if $c_1 =0$, then either 0 also occurs, or -1 occurs at least twice. 
\end{theorem}
\begin{proposition}\cite[Hartshorne, Proposition 5.1]{Hartshorne1980}\label{alpinista4}
Let $\mathcal{F}$ be a semistable rank 2 reflexive sheaf on $\mathbb{P}^3$ with $c_1=0$ or $-1$. Let $k= max \{-k_i \}$ as $k_i$ runs over the spectrum of $\mathcal{F}$. Assume there is a $r_0$ with $-k < r_0 \leq -2$ which occurs only once in the spectrum. Then each $k_i$ with $-k \leq k_i \leq r_0$ occurs exactly once in the spectrum
\end{proposition}
Now, let $\mathcal{F}$ be a rank 2 locally free sheaf with $c_1=-1$, $H^0(\mathcal{F}(-1))=0$ and $\chi$ its spectrum. If $k_j \in \chi$, by symmetry $-k_j -1 $ occurs in $\chi$. \\
Define $p_j=\{-j-1,j\}$ for each integer. We will say that $p_j$ occurs in $\chi$ if $-j-1$ and $j$ are in $\chi$. Note that $p_j = p_{-j-1}$ so we will only deal with $j \geq 0$. \\
For example, $p_0 p_1^2$ denotes the spectrum $\{-2^2,-1,0,1^2  \}$.\\
By Theorem (\ref{alpinista3}), if $p_j$ occurs in $\chi$ and $j \geq 1$ then every $p_l$ with $1 \leq l \leq j$ also occurs, if $\mathcal{F}$ is stable then every $p_l$ with $0 \leq l \leq j$ occurs. \\
With this in mind, given an integer $c_2$ we can write a list of possible spectra a stable bundle satisfying the hypothesis of (\ref{alpinista1}) can have:
\begin{center}
 \begin{tabular}{||c | c||} 
 \hline
 $c_2$ & $\chi$ \\ [0.5ex] 
 \hline\hline
 \\[-1em]
 1 & $\emptyset$ \\ 
 \hline
 2 & $p_0$ \\
 \hline
 3 & $\emptyset$ \\
 \hline
 4 & $p_0 ^2,p_0 p_1$ \\
 \hline
 5 & $\emptyset$  \\
 \hline
 6 & $p_0^3,p_1p_0^2,p_1^2 p_0, p_2 p_1 p_0$ \\
 \hline
 7 & $\emptyset$ \\
 \hline
 8 & $p_0^4,p_1p_0^3,p_1^2p_0^2,p_1^3p_0,p_2p_1^2p_0,p_2p_1p_0^2,p_3p_2p_1p_0$ \\ [1ex] 
 \hline
\end{tabular}
\end{center}
Notice that we don't have $p_2^2p_1p_0$ in the line $c_2=8$, because Proposition (\ref{alpinista4}) tells us this case cannot happen. To see this notice that $max \{-k_i\}_{i=1,...,8}=3$ and take $r_0=-2$, we have $-3=max \{k_i\}_{i=1,...,8} <r_0=-2$ and $-2$ occurs only once in the spectrum, so Proposition (\ref{alpinista4}) states that any $k_i$ such that $-3=-k \leq k_i \leq -2$ must occur only once in the spectrum, this includes $-3$. But $-3$ appears twice in $\chi =p_2^2 p_1 p_0$ so $\chi$ can't happen.
Since any integer in $\chi$ must have another different integer associated with it we see there is no rank 2 locally free sheaf with $c_1=-1$ satisfying $H^0(\mathcal{F}(-1))=0$ with odd $c_2$. \\
It is interesting to notice this is not the case for $c_1=0$. In this case 0 is not associated with any other integer, so for instance it is possible to have a spectrum $\chi = \{-1,0^3,1 \}$. \\
A general formula for any $c_2$ is given by $\prod_{i=1}^{\frac{c_2}{2}} \ p_i^{k_i}$ with: \\
(i) $\Sigma_{i=1}^{\frac{c_2}{2}} \ k_i=c_2$ \\
(ii) If $p_k \in \chi$ then $p_j \in \chi$ for every $0\leq j \leq k$ \\
(iii) If $p_k \in \chi$ is such that $k= max \{l \ | \ p_l \in \chi \}$ and $p_j$ occurs only once with $p_1 \leq p_j < p_k$ then any $p_{j'}$ with $p_j \leq p_{j'} \leq p_k$ occurs only once

Theorem (\ref{Horrocks1}) tells us that every locally free sheaf of rank 2 in $\mathbb{P}^3$ is the cohomology of a monad of the form
\begin{equation}
\mathcal{V}_0 \to \mathcal{V}_1 \to \mathcal{V}_2
\end{equation}
where $\mathcal{V}_0, \mathcal{V}_1, \mathcal{V}_2$ are sums of line bundles. Such monad can be chosen to be minimal in the sense that it is built from a minimal resolution of the first cohomology module \cite[Rao]{RAO198423} and \cite[Decker]{Decker1990}.\\
\iffalse
If $\mathcal{E}$ is a bundle of rank 2 with $c_1=-1$ then $\mathcal{E}^* \cong \mathcal{E}(1)$ \cite[Hartshorne, Proposition 1.10]{Hartshorne1980}. \\

Suppose 
\begin{equation}\label{monad1}
\bigoplus_{i=1}^{r} \mathcal{O}_{\mathbb{P}^3}(a_i) \to \bigoplus_{i=1}^{s} \mathcal{O}_{\mathbb{P}^3}(b_i) \to \bigoplus_{i=1}^{t} \mathcal{O}_{\mathbb{P}^3}(c_i)
\end{equation}
is a monad for $\mathcal{E}$. Dualize and twist it by -1 to get a monad
\begin{equation}
\bigoplus_{i=1}^{t} \mathcal{O}_{\mathbb{P}^3}(-c_i-1) \to \bigoplus_{i=1}^{s} \mathcal{O}_{\mathbb{P}^3}(-b_i-1) \to \bigoplus_{i=1}^{s} \mathcal{O}_{\mathbb{P}^3}(-a_i-1)
\end{equation}
which has $\mathcal{E}^* (-1) \cong \mathcal{E}$ as its cohomology. Using \cite[Okonek, Schneider \& Spindler, Corollary 1 p.279]{Okonek} we can write (\ref{monad1}) as
~~~~~~~~~~~~~~~~~~~~~~~~~
Let $\mathcal{E}$ be a vector bundle of rank 2 on $X=\mathbb{P}^3$ with $c_1=-1$ and $\{g_i \}_{i \in \Lambda}$ be a minimal set of generators of $H^1_*(\mathcal{E})$ where $g_i \in H^1(\mathcal{E}(-l_i))$. Note that $\omega_X\cong\mathcal{O}_{\mathbb{P}^3}(-4)$, then $H^1_*(\mathcal{E}^* \otimes \omega_X) \cong H^1_*(\mathcal{E})$ since $\mathcal{E}^* \cong \mathcal{E}(1)$. Now, taking $h_i:=g_i$ and using Theorem (\ref{Horrocks1}) we obtain a Horrocks monad of the form
\begin{equation}
\mathcal{V}_\bullet : \bigoplus\limits_{i=1}^r \omega_X (k_i) \overset{ \alpha }{ \to} \mathcal{F} \overset{ \beta }{ \to} \bigoplus\limits_{j=1}^s \mathcal{O}_X (l_j)
\end{equation}
To compute $k_i$ we just need to see that $g_i \in H^1(\mathcal{E}(-l_i)) \cong H^1(\mathcal{E}^* \otimes \omega_X (-l_i+3))$ so $k_i=-l_i+3$. Finally we use Horrocks criterion to show that $\mathcal{F}$ is a sum of line bundles since $H^1_*(\mathcal{F})=H^2_*(\mathcal{F})=0$ \cite[Horrocks]{doi:10.1112/plms/s3-14.4.689}.
\fi
In \cite[Decker, Proposition 1]{Decker1990}, Decker shows that $\mathcal{E}$ is the cohomology of a monad of the form
\begin{equation}\label{finalhorrocks}
\bigoplus_{i=1}^{r} \mathcal{O}_{\mathbb{P}^3}(-c_i-1) \to \bigoplus_{i=1}^{r+1} (\mathcal{O}_{\mathbb{P}^3}(-b_i-1) \oplus \mathcal{O}_{\mathbb{P}^3}(b_i)) \to \bigoplus_{i=1}^{r} \mathcal{O}_{\mathbb{P}^3}(c_i)
\end{equation}
Note that (\ref{finalhorrocks}) is Horrocks, hence by Theorem (\ref{Horrocks1}) $r$ coincides with the number of generators of $H^1_* (\mathcal{E})$ as a graded module over the ring of polynomials in four variables, while $-c_i$ are the degrees of these generators.
\section{The Hartshorne-Serre correspondence}
In \cite[Hartshorne]{hartshorne1974}, Hartshorne, following the ideas of previous works of Horrocks \cite[Horrocks, 1968]{HORROCKS1968117} and Serre \cite[Serre, 1960]{serre_2003} proved that a nonsingular subvariety $Y$ of codimension $2$ in $\mathbb{P}^n$ with $n \geq 3$ occurs as the zero-set of a section of a bundle $\mathcal{E}$ of rank 2 on $\mathbb{P}^n$ if and only if its canonical sheaf $\omega_Y$ is a multiple of the hyperplane sheaf $\mathcal{O}_Y(1)$ (if this last condition is satisfied we say $Y$ is \textbf{subcanonical}). \\
In \cite[Hartshorne]{Hartshorne1980} Hartshorne provided a version of the correspondence between rank 2 reflexive sheaves on $\mathbb{P}^3$ and arbitrary curves in $\mathbb{P}^3$.

%Recall that a projective scheme $Y \to \mathbb{P}^n$ of codimension $r$ in $\mathbb{P}^n$ is a complete intersection if and only if (equivalently there are $r$ hypersurfaces $H_1,...,H_r$ such that $Y=H_1 \cap ... \cap H_r$ as schemes).
Recall that a projective variety $Y \to \mathbb{P}^n$ of codimension $r$ is a local complete intersection if the ideal sheaf can be locally generated by $r$ elements at every point.
\begin{theorem}\cite[Hartshorne, Theorem 1.1]{Hartshorne1978}\label{hartshorneserre}
Fix a line bundle $\mathcal{L}$ on $\mathbb{P}^3$, there is a bijective correspondence between \\
(i) the set of triples $(\mathcal{E},s,\phi)$ modulo equivalence relation $\sim$, where $\mathcal{E}$ is a vector bundle of rank 2 on $\mathbb{P}^3$, $s \in H^0(\mathcal{E})$ is a global section whose scheme of zeros $Y$ has codimension 2, $\phi : \wedge^2 \mathcal{E} \to \mathcal{L}$ is an isomorphism and $(\mathcal{E},s,\phi) \sim (\mathcal{E}', s', \phi ')$ if and only if there is an isomorphism $\psi:\mathcal{E}\to \mathcal{E}'$ and an element $\lambda \in k-\{0\}$ such that $s'=\lambda\psi(s)$ and $\phi'=\lambda^2 \phi \circ (\wedge^2 \psi)^{-1}$. \\
(ii) the set of pairs $(Y,\xi)$ where $Y$ is a locally complete intersection curve in $\mathbb{P}^3$ and $\xi : \mathcal{L} \otimes \omega_{\mathbb{P}^3} \otimes \mathcal{O}_Y \to \omega_Y$ is an isomorphism.
\end{theorem}
%\textcolor{red}{professor, esqueci de mencionar: tentei incluir a demonstração da correspondencia, mas tive problema em umas partes. Então por enquanto deixei por fora, só enunciando mesmo o teorema, mas de qualquer jeito vou deixar aparecendo nesse pdf para ficar claro  onde tive problema}\\
%----------------------------- \\
%\textcolor{red}{aqui começa a parte que deixei por fora}:\\
\iffalse
Before proving the theorem, we present an important lemma
\begin{lemma}\cite[Serre, p.2-08]{serre_2003}
Let $M$ be a $A$-module with $A$ a ring. Suppose $M$ has homological dimension $dh(M) \leq 1$ and let $\xi \in Ext^1_A(M,A)$. Let $E_\xi$ be the corresponding extension of $M$ given by the exact sequence $0 \to A \to E_\xi \to M \to 0$. Then $E_\xi$ is projective if and only if $\xi$ generates $Ext^1_A(M,A)$.
\end{lemma}
\begin{proof}
Notice that $dh(E_\xi) \leq 1$, by (\ref{homologicaldimension}) we have $Ext^1_A(E_\xi,A)=0$ if and only if $E_\xi$ is projective. Applying the functor $Hom_A(-,A)$ we have $Hom_A(A,A) \to Ext^1_A(M,A) \to Ext^1_A(E_\xi,A) \to 0$. In other words $A \overset{.\xi}{\to} Ext^1_A(M,A) \to Ext^1_A(E_\xi,A) \to 0$ where the map $.\xi$ denotes the multiplication $x \mapsto x\xi$, and such map is surjective if and only if $\xi$ generates $Ext^1_A(M,A)$.
\end{proof}
We will now prove theorem (\ref{hartshorneserre}). \textcolor{red}{professor, esqueci de mencionar, tive problema para entender as seguintes sentenças em vermelho na demonstração}
\begin{proof}
Let $(\mathcal{E},s,\phi)$ be such triple and $Y$ the scheme of zeros of $s$. Since $Y$ has codimension $2$, locally $2$ generators of $\mathcal{I}_Y$ form a regular sequence, so the local Koszul complexes gives a resolution of $\mathcal{I}_Y$ (\cite[Hartshorne, proof of II.7.11]{hartshorne_2010}):
\begin{equation}\label{koszulhartshorneserreeq}
0 \to \bigwedge^2\mathcal{E}^* \to \mathcal{E}^* \overset{s}{\to} \mathcal{I}_Y \to 0
\end{equation}
Using the isomorphism $\phi$ we can rewrite the sequence
\begin{equation}
0 \to \mathcal{L}^* \to \mathcal{E}^* \overset{s}{\to} \mathcal{I}_Y \to 0
\end{equation}
Which determines an element $\xi \in Ext^1(\mathcal{I}_Y,\mathcal{L}^*)$. Now consider
\begin{equation}
0 \to \mathcal{I}_Y \to \mathcal{O}_{\mathbb{P}^3} \to \mathcal{O}_Y \to 0
\end{equation}
Apply the functor $Hom(-,\mathcal{L}^*)$:
\begin{equation}
H^1(\mathcal{L}^*)=Ext^1(\mathcal{O}_{\mathbb{P}^3},\mathcal{L}^*) \to Ext^1(\mathcal{I}_Y,\mathcal{L}^*) \to Ext^2(\mathcal{O}_Y, \mathcal{L}^*) \to Ext^2(\mathcal{O}_{\mathbb{P}^3},\mathcal{L}^*)=H^2(\mathcal{L}^*)
\end{equation}
Using that \textcolor{red}{$H^i(\mathcal{L}^*)=0$ for $i=1,2$} yields the isomorphism $Ext^1(\mathcal{I}_Y,\mathcal{L}^*) \cong Ext^2(\mathcal{O}_Y,\mathcal{L}^*)$. The local to global spectral sequence \textcolor{red}{gives the isomorphism $Ext^2(\mathcal{O}_Y,\mathcal{L}^*)\cong H^0(\mathcal{E}xt^2(\mathcal{O}_Y,\mathcal{L}^*))$}. By definition $\omega_{Y}=\mathcal{E}xt^2_{\mathbb{P}^3}(\mathcal{O}_Y,\omega_{\mathbb{P}^3})$ so $H^0(\mathcal{E}xt^2(\mathcal{O}_Y,\mathcal{L}^*)) \cong H^0(Y,\omega_{Y} \otimes \omega_{\mathbb{P}^3}^* \otimes \mathcal{L}^*) \cong Hom(\omega_{\mathbb{P}^3} \otimes \mathcal{L} \otimes \mathcal{O}_Y, \omega_{Y})$. Hence $\xi$ can be interpreted as a morphism $\xi : \omega_{\mathbb{P}^3}\otimes\mathcal{L} \otimes \mathcal{O}_Y \to \omega_Y$, we want to show this morphism is an isomorphism. Let $U$ be an open affine subset of $\mathbb{P}^3$ such that $\mathcal{E}|_U$ is free. \textcolor{red}{$Y \cap U$ is a complete intersection} and the corresponding sequence (\ref{koszulhartshorneserreeq}) is a Koszul complex. Using the preceding lemma and the fact that $\mathcal{E}^*$ is free on $U$ we get that $\xi$ is an isomorphism \textcolor{red}{preciso terminar a volta, mas como fiquei travado nos problemas em vermelho não fiz ainda}.
\end{proof}
------------------
\textcolor{red}{aqui termina a parte que deixei por fora}
\fi
If $Y$ is a subcanonical curve with $\omega_Y \cong \mathcal{O}_Y(a)$ we can take $\mathcal{L}=\mathcal{O}_{\mathbb{P}^3}(a+4)$, thus $\omega_Y \cong \mathcal{L}\otimes\omega_{\mathbb{P}^3}\otimes \mathcal{O}_Y$. The theorem then gives a bundle $\mathcal{E}$, a section $s$ and an isomorphism $det(\mathcal{E}) \cong \mathcal{L}$. Since $c_1(det(\mathcal{E}))=c_1(\mathcal{E})$ \cite[Hartshorne, p.430]{hartshorne_2010} we get $c_1(\mathcal{E})=a+4$. We also have $c_2(\mathcal{E})=[Y]=deg(Y)[H]^2$  \cite[Hartshorne, C6 p.431]{hartshorne_2010} so the second Chern class corresponds to the degree of $Y$.
From the proof of theorem \ref{hartshorneserre} we get an exact sequence:
\begin{equation}
0 \to \mathcal{L}^* \to \mathcal{E}^* \to \mathcal{I}_Y \to 0
\end{equation}
Tensor it by $\mathcal{O}_{\mathbb{P}^3}(a+4)$ and define $\mathcal{F}=\mathcal{E}^* (a+4)$.
\begin{equation}\label{hartshorneserreeq}
0 \to \mathcal{O}_{\mathbb{P}^n} \to \mathcal{F} \to \mathcal{I}_Y(a+4) \to 0
\end{equation}
The Chern classes of $\mathcal{F}$ are $c_1(\mathcal{F})=-(a+4)+2(a+4)=c_1(\mathcal{E})$ and $c_2(\mathcal{F})=c_2(\mathcal{E})-(a+4)c_1(\mathcal{E})+(a+4)^2=c_2(\mathcal{E})$.\\So from now on the \textbf{vector bundle $\mathcal{E}$ corresponding to a subcanonical curve $Y$} will be the vector bundle satisfying sequence (\ref{hartshorneserreeq}).\\
%\begin{theorem}[\textcolor{red}{ref stable vector bund of rank 2 on P3 thm 1.5}]
%Let $X$ be a locally complete intersection curve in $\mathbb{P}^3$ with $H^0(\mathcal{O}_X)=k$. Let $L$ be an invertible sheaf on $X$, let $u: I_X \to L \to 0$ be a surjective map, and let $Y$ be the scheme defined by $I_Y = ker \ u$. Let $m$ be an integer. Then the following conditions are equivalent: \\
%(i) $\omega_Y \cong \mathcal{O}_Y (-m)$ \\
%(ii) $L \cong \omega_X(m)$ and the map \\
%$\bar{u}:H^1(I_X (-m)) \to H^1(\omega_X)$\\
%induced by $u$ is the zero map. \\
%Furthermore, if $m \geq 0$ and $\omega_X \otimes (I_X/I_X^2)^* (m)\\
%$ is generated by global sections, then there is a map $u : I_X \to \omega_X(m) \to 0$ satisfying the condition (ii). In particular, the corresponding $Y$ satisfies (i).
%\end{theorem}
%Both theorems combined give 
%\begin{corollary}[\textcolor{red}{ref stable vect bund of rank 2 on P3 cor 1.6}]
%For any locally complete intersection curve $Y$ in $\mathbb{P}^3$ with $H^0(\mathcal{O}_X)=k$, there is a scheme $Y$ with support equal to $X$, which corresponds to a vector bundle of rank 2 on $\mathbb{P}^3$.
%\end{corollary}
Now, let $\mathcal{T}$ be the tangent bundle of $\mathbb{P}^3$. From the sequence $0 \to \mathcal{O}_{\mathbb{P}^3} \to \mathcal{O}_{\mathbb{P}^3}(1)^{4} \to \mathcal{T} \to 0$ we obtain $td(\mathcal{T})=td(\mathcal{O}_{\mathbb{P}^3}(1)^{4})$ where $td$ is the Todd class, so $td(\mathcal{T})=1+2H+\frac{11}{6}H^2+H^3$. Let $\mathcal{F}$ be a rank $2$ bundle on $\mathbb{P}^3$ with Chern classes $c_1(\mathcal{F})=c_1$ and $c_2(\mathcal{F})=c_2$. The Chern character $ch(\mathcal{F})$ is $2+c_1H+\frac{c_1^2-2c_2}{2}+\frac{c_1^3-3c_1 c_2}{6}$. Multiplying $td(\mathcal{T})$ with $ch(\mathcal{F})$ and summing the degree $3$ terms gives $2+\frac{11c_1}{6}+(c_1^2-2c_2)+\frac{c_1^3-3c_1c_2}{6}$. Rearranging terms and using the Riemann-Roch theorem gives
\begin{equation}\label{riemannroch}
\chi^e(\mathcal{F})={{c_1+3}\choose{3}}-2c_2-\frac{c_1c_2}{2}+1
\end{equation}
We will give a few examples of bundles corresponding to curves
\begin{example}[Disjoint union of conics]\label{exemplobeilinson}
Let $Y=\bigsqcup_{i=1}^r C_i$ be a disjoint union of r conics in $\mathbb{P}^3$ and $i$ the inclusion morphism. We have $deg(Y)=2r$ and $ \omega_Y \cong  \mathcal{O}_Y(-1)$. Let $\mathcal{E}'$ be the vector bundle corresponding to $Y$. We have an exact sequence $0 \to \mathcal{O}_{\mathbb{P}^3} \to \mathcal{E'} \to \mathcal{I}_Y(3) \to 0$ with $c_1(\mathcal{E'})=3$ and $c_2(\mathcal{E'})=2r$. Let $\mathcal{E}=\mathcal{E'}(-2)$, then $c_1(\mathcal{E})=-1$ and $c_2(\mathcal{E})=2r-2$. The sequence
\begin{equation}\label{seqconica}
0 \to \mathcal{O}_{\mathbb{P}^3}(-2) \to \mathcal{E} \to \mathcal{I}_Y(1) \to 0
\end{equation}
shows that if $r \geq 2$ then $H^0(\mathcal{E})=0$ since $Y$ is not contained in any hyperplane (this implies $H^0(\mathcal{I}_Y(1))=0$ ). Since $\mathcal{E}$ is normalized it means $\mathcal{E}$ is stable. Notice $\mathcal{E}^* \cong \mathcal{E}(1)$ because $c_1(\mathcal{E})=-1$.\\
Now, since $H^0(\mathcal{E}(k))=0$ for every $k \leq 0$ using Serre duality we have $H^3(\mathcal{E}(-k-3))=0$ for $k \leq 0$. In other words $H^3(\mathcal{E}(k))=0$ for $k \geq -3$.\\
From sequence (\ref{seqconica}) twisted by $k \in \mathbb{Z}$ we get $H^1(\mathcal{E}(k)) \cong H^1(\mathcal{I}_Y(k+1))$. From $$0 \to \mathcal{I}_Y(k) \to \mathcal{O}_{\mathbb{P}^n}(k) \to \mathcal{O}_Y(k) \to 0$$ since $H^0(\mathcal{O}_Y(k))=\bigoplus_{i=1}^r H^0(\mathcal{O}_{C_i}(k))=0$ for $k \leq -1$ we get $H^1(\mathcal{I}_Y(k))=0$ for $k \leq -1$, hence $H^1(\mathcal{E}(k))=0$ for $k \leq -2$. By Serre duality $H^1(\mathcal{E}(k)) \cong H^2(\mathcal{E}(-k-3))^*$ so $H^2(\mathcal{E}(k))=0$ for $k \geq -1$.\\
\iffalse
Using theorem \ref{beilinson1} on $\mathcal{E}(-1)$ gives a spectral sequence with first sheet
\begin{center}
\begin{tikzpicture}
  \matrix (m) [matrix of math nodes,
    nodes in empty cells,nodes={minimum width=5ex,
    minimum height=5ex,outer sep=-5pt},
    column sep=2ex,row sep=3ex]{    
    &            &             &            & q     &   \\
     &     \mathcal{E}^{-3,3}_1   &     0     &    0       &  0    &   \\
     &  0  & \mathcal{E}^{-2,2}_1  & \mathcal{E}^{-1,2}_1  &   0   &   \\
     &     0     & 0 &  0  & \mathcal{E}^{0,1}_1  \\
     &     0    &  0         &  0        &   0   &  p \\ 
  \quad\strut  &  &    &   &   &  \strut \\};
\draw[->]  (m-6-5.east) -- (m-1-5.east) ;
\draw[->] (m-5-1.south) -- (m-5-6.south) ;
\end{tikzpicture}
\end{center}
Where $\mathcal{E}^{p,q}_1=H^q(\mathcal{E}(p-1))\otimes\Omega_{\mathbb{P}^3}^{-p}(-p)$
\textcolor{red}{Professor, a gente tava discutindo sobre o termo $E^{-2,2}$ ser 0 argumentando sobre $H^1(\mathcal{I}_Y(2))$ mas não precisa fazer isso pois $E^{-2,2}$ tem o feixe $\Omega^4(4)$ que é 0 certo? Aí acho que não precisa argumentar de pegar a união disjunta de conicas fora de uma quádrica.}\\
The second sheet is
\begin{center}
\begin{tikzpicture}
  \matrix (m) [matrix of math nodes,
    nodes in empty cells,nodes={minimum width=5ex,
    minimum height=5ex,outer sep=-5pt},
    column sep=2ex,row sep=3ex]{    
    &            &             &            & q     &   \\
     &     \mathcal{E}^{-3,3}_1     &     0       &    0       &  0    &   \\
     &  0  & ker(d^{-2,2}_1)  & coker(d^{-2,2}_1)  &   0   &   \\
     &     0     & 0 &  0  & \mathcal{E}^{0,1}_1  \\
     &     0    &  0         &  0        &   0   &  p \\ 
  \quad\strut  &  &    &   &   &  \strut \\};
\draw[->]  (m-6-5.east) -- (m-1-5.east) ;
\draw[->] (m-5-1.south) -- (m-5-6.south) ;
\end{tikzpicture}
\end{center}
And the third sheet
\begin{center}
\begin{tikzpicture}
  \matrix (m) [matrix of math nodes,
    nodes in empty cells,nodes={minimum width=5ex,
    minimum height=5ex,outer sep=-5pt},
    column sep=2ex,row sep=3ex]{    
    &            &             &            & q     &   \\
     &     ker(d^{-3,3}_2)     &     0       &    0       &  0    &   \\
     &  0  & ker(d^{-2,2}_2)  & coker(d^{-2,2}_1)  &   0   &   \\
     &     0     & 0 &  0  & coker(d^{-2,2}_2)  \\
     &     0    &  0         &  0        &   0   &  p \\ 
  \quad\strut  &  &    &   &   &  \strut \\};
\draw[->]  (m-6-5.east) -- (m-1-5.east) ;
\draw[->] (m-5-1.south) -- (m-5-6.south) ;
\end{tikzpicture}
\end{center}
Finally the fourth sheet
\begin{center}
\begin{tikzpicture}
  \matrix (m) [matrix of math nodes,
    nodes in empty cells,nodes={minimum width=5ex,
    minimum height=5ex,outer sep=-5pt},
    column sep=2ex,row sep=3ex]{    
    &            &             &            & q     &   \\
     &     ker(d^{-3,3}_3)     &     0       &    0       &  0    &   \\
     &  0  & ker(d^{-2,2}_2)  & coker(d^{-2,2}_1)  &   0   &   \\
     &     0     & 0 &  0  & coker(d^{-3,3}_3)  \\
     &     0    &  0         &  0        &   0   &  p \\ 
  \quad\strut  &  &    &   &   &  \strut \\};
\draw[->]  (m-6-5.east) -- (m-1-5.east) ;
\draw[->] (m-5-1.south) -- (m-5-6.south) ;
\end{tikzpicture}
\end{center}
Every morphism is zero, so we see that $E^{p,q}_4=E^\infty$. By theorem \ref{beilinson1} we have $coker(d^{-3,3}_3)=coker(d^{-2,2}_1)=0$ and $ker(d^{-2,2}_2)=\mathcal{E}(-1)$. But if $E^{0,1}_1=0$ we have $E^{p,q}_2=E^\infty$ so $ker(d^{-2,2}_1)=\mathcal{E}(-1)$, thus yielding an exact sequence:
\begin{equation}
 0 \to \mathcal{E}(-1) \to H^2(\mathcal{E}(-3)) \otimes \Omega^2(2) \to H^2(\mathcal{E}(-2)) \otimes \Omega(1) \to 0
\end{equation}
%We can use this to compute $h^2(\mathcal{E}(-3))$ and $h^2(\mathcal{E}(-2))$:\\
%$c_1(\mathcal{E}(k))=c_1(\mathcal{E})+2k$ and $c_2(\mathcal{E}(k))=c_2(\mathcal{E})+c_1(\mathcal{E})k+k^2$ thus $c_1(\mathcal{E}(-3))=-7$, $c_2(\mathcal{E}(-3))=2r+10$, $c_1(\mathcal{E}(-2))=-5$ and $c_2(\mathcal{E}(-2))=2r+4$. Applying (\ref{riemannroch}) we obtain $h^2(\mathcal{E}(-3))=3r+16$ and $h^2(\mathcal{E}(-2))=r+3$. Thus exact sequence becomes
%\begin{equation}
%0 \to \mathcal{E}(-1) \to\Omega^2(2)^{\oplus (3r+16)} \to \Omega(1)^{\oplus (r+3)} \to 0
%\end{equation}
\fi
We wish to compute the spectrum of $\mathcal{E}$. Notice that the number of $p_i$ in the spectrum where $i \geq 1$ is zero since $\# p_i=h^1(\mathcal{E}(-1-i))-2h^1(\mathcal{E}(-2-i))+h^1(\mathcal{E}(-3-i))$ for $i \geq 1$. So $\chi(\mathcal{E})$ can only have $p_0$'s, which implies $\chi(\mathcal{E})=p_0^{\frac{2r-2}{2}}=p_0^{r-1}$.\\
We could also use formula (\ref{spectrum}): Since
$H^i(\mathcal{E}(-1))=0$ for $i \not= 1$ we can use equation (\ref{riemannroch}) to compute $h^1(\mathcal{E}(-1))$. 
Since $c_1(\mathcal{E}(-1))=-1-2=-3$ and $c_2(\mathcal{E}(-1))=2r-2+(-1)(-1)+1=2r$, using formula (\ref{riemannroch}) we get $h^1(\mathcal{E}(-1))=r-1$, hence $\# p_0=r-1$, in other words $\chi(\mathcal{E})=p_0^{r-1}$.
\end{example}
\begin{example}\label{exemplobeilinson2}
Let $m \geq 2$ be an integer, $Y_1$ a plane curve of degree $2m-2$ and $Y_2$ a complete intersection of surfaces of degree $m$ and $m-1$. Let $Y=Y_1 \bigsqcup Y_2$. Since $\omega_{Y_i} \cong \mathcal{O}_{Y_i}(2m-5)$ we have $\omega_{Y} \cong \mathcal{O}_Y(2m-5)$ with $deg(Y)=m(m-1)+2m-2=m^2+m-2$.\\
The correspondence gives a vector bundle $\mathcal{E'}$ satisfying
\begin{equation}
0 \to \mathcal{O}_{\mathbb{P}^n} \to \mathcal{E'} \to \mathcal{I}_Y(2m-1) \to 0
\end{equation}
We wish to normalize $\mathcal{E}'$, so consider $\mathcal{E}=\mathcal{E}'(-m)$. Then $c_1(\mathcal{E})=2m-1+2(-m)=-1$ and $c_2(\mathcal{E})=deg(Y)+(2m-1)(-m)+m^2=2m-2$. So the new exact sequence is
\begin{equation}
0 \to \mathcal{O}_{\mathbb{P}^n}(-m) \to \mathcal{E} \to \mathcal{I}_Y(m-1) \to 0
\end{equation}
$\mathcal{E}$ is stable because $Y$ is not contained in any surface of degree $m-1$. $Y$ has two components so from the exact sequence $0 \to H^0(\mathcal{O}_{\mathbb{P}^n}) \to H^0(\mathcal{O}_Y) \to H^1(\mathcal{I}_Y) \to 0$ we get $h^1(\mathcal{I}_Y)=1$, which implies $H^1(\mathcal{E}(-m+1)) \not=0$. But $c_2=2m-2$ so $H^1(\mathcal{E}(-\frac{c_2}{2})) \not=0$ which implies $\chi(\mathcal{E})=p_0p_1...p_{\frac{c_2-2}{2}}$.
\end{example}
%Next example gives us a monad for $\mathcal{E}$
%\begin{example}
%Consider the same setup as in example (\ref{exemplobeilinson}). This time we are going to use theorem (\ref{beilinson2}). For $1 \leq p \leq 3$ we have an exact sequence
%\begin{equation}
%0 \to \Omega^p(p) \to \mathcal{O}_{\mathbb{P}^n}^{\oplus{{4}\choose{p}}} \to \Omega^{p-1}(p) \to 0
%\end{equation}
%Applying $\otimes \mathcal{E}$ gives
%\begin{equation}
%0 \to \mathcal{E}\otimes\Omega^p(p) \to \mathcal{E}^{\oplus{{4}\choose{p}}} \to \mathcal{E}\otimes\Omega^{p-1}(p) \to 0
%\end{equation}
%Since $H^0(\mathcal{E}(k))=0$ for $k \leq 0$ we have $H^0(\mathcal{E}\otimes\Omega^p(p+k))=0$ for $k \leq 0$
%\end{example}
\begin{example}\label{exemplobeilinson3}
Let $Y_1$ be the disjoint union of two plane curves of degree 4 and $Y_2$ be a curve from the intersection of two surfaces of degree 3 and 2. We have $\omega_{Y_i}=\mathcal{O}_{Y_i}(1)$. Let $\mathcal{E}_i'$ be the vector bundle corresponding to $Y_i$. 
\begin{equation}
0 \to \mathcal{O}_{\mathbb{P}^n} \to \mathcal{E}_i' \to \mathcal{I}_{Y_i}(5) \to 0
\end{equation}
$c_1(\mathcal{E}_i')=5$ and $c_2(\mathcal{E}_i')=deg(Y_i)$. Set $\mathcal{E}_i=\mathcal{E}_i'(-3)$ so that $c_1(\mathcal{E}_i)=-1$, $c_2(\mathcal{E}_1)=2$ and $c_2(\mathcal{E}_2)=0$.
\begin{equation}\label{ex6}
0 \to \mathcal{O}_{\mathbb{P}^n}(-3) \to \mathcal{E}_i \to \mathcal{I}_{Y_i}(2) \to 0
\end{equation}
It only makes sense to talk about the spectrum of $\mathcal{E}_1$ since the spectrum is defined only if $c_2 > 0$:\\
The number of ones in the spectrum is given by $\# 1 = h^1(\mathcal{E}_1(-2))-2h^1(\mathcal{E}_1(-3))+h^1(\mathcal{E}_1(-4))$. From (\ref{ex6}) we have $H^1(\mathcal{E}_1(k)) \cong H^1(\mathcal{I}_{Y_1}(k+2))$. We already know that $H^1(\mathcal{I}_{Y_1}(k))=0$ for $k<0$ since $Y_1$ is a disjoint union of complete intersections, indeed if $Z$ is a complete intersection curve of surfaces of degree $d_1$ and $d_2$ then using the exact sequence $$0 \to \mathcal{O}_{\mathbb{P}^n}(-d_1-d_2) \to \mathcal{O}_{\mathbb{P}^n}(-d_2) \oplus \mathcal{O}_{\mathbb{P}^n}(-d_1) \to \mathcal{I}_Z \to 0$$ we conclude $H^1(\mathcal{I}_{Z}(k))=0$ for $k<0$. Looking at the exact sequence $$0 \to \mathcal{I}_{Y_1} \to \mathcal{O}_{\mathbb{P}^n} \to \mathcal{O}_{Y_1} \to 0$$ and noticing that $H^0(\mathcal{I}_{Y_1})=0$ and $h^0(\mathcal{O}_{Y_1})$ is the number of connected components, we get $h^1(\mathcal{I}_{Y_1})=1$. So the spectrum of $Y_1$ is $\chi(\mathcal{E}_1)=p_1$. In particular this implies that $\mathcal{E}_1$ is not stable (Theorem \ref{alpinista3}), hence $Y_1$ is contained in a surface of degree 2.\\
Now consider $Y=Y_1 \bigsqcup Y_2$. Let $\mathcal{E}$ be the normalized vector bundle corresponding to $Y$.$Y$ is not contained in any surface of degree $2$ so $\mathcal{E}$ is stable. We have $c_1(\mathcal{E})=-1$ and $c_2(\mathcal{E})=10$ Using the same argument as before $\# 1 = 2$ so $p_1^2$ appears in $\chi(\mathcal{E})$. We can only complete the spectrum with $p_0$ because $h^1(\mathcal{E}(k))=0$ for $k \leq -3$ so $\chi(\mathcal{E})=p_0^3p_1^2$. We could keep on adding those curves in order to generate other stable vector bundles with only $p_1$ and $p_0$.
\end{example}
We now have three families of curves, each generating a family of vector bundles with a specific spectrum, it is natural to ask what kind of monad we can get from each family. We would like to use Theorem (\ref{Horrocks1}), but first we need to find the amount of minimal generators of $H^1_*(\mathcal{E})$ and their respective degrees.\\
Let $\mathcal{E}$ be a rank 2 stable vector bundle on $\mathbb{P}^3$ with $c_1=-1$ and call $M:=H^1_*(\mathcal{E})$ the first cohomology module over $S=k[x_0,x_1,x_2,x_3]$ with $M_j$ being the degree $j$th submodule of $M$. Given $l\in \mathbb{Z}$ define $\rho(l)$ to be amount of minimal generators for $M$ in degree $l$ and let $m_j:=h^1(\mathcal{E}(j))=dim(M_j)$. It has a spectrum of the form $\chi(\mathcal{E})= \{(-k-1)^{s(k)},...,-1^{s(0)},0^{s(0)},...,k^{s(k)} \}$, since $s(k+1) = 0$, $s(k) \not=0$ and $s(l) =h^1(\mathcal{E}(-l-1))-2h^1(\mathcal{E}(-l-2))+h^1(\mathcal{E}(-l-3)) $ we have that $M_{-k-1}$ is the lowest degree part of $M$ such that it is nonzero. In particular this implies $\rho(-k-1)=m_{-k-1}$, using the fact that $m_l=\# \{k_j \in \chi(\mathcal{E}) | k_j \geq l-1 \}$ gives:
\begin{equation}\label{maxmod}
\rho(-k-1)=s(k)
\end{equation}
In other words, if we know how many times the greatest integer $k$ appears in $\chi(\mathcal{E})$, then we know the amount of minimal generators in degree $-k-1$ for $M$, which gives a summand of our monad. Following the notation of equation (\ref{finalhorrocks}) the number $k+1$ will be the largest number between the $c_i$'s.\\
We still need to find the other $c_i < k+1$, the following proposition given by Hartshorne is useful
\begin{proposition}\cite[Hartshorne \& Rao, p.796]{hartshorne1991}\label{desigualdade}
For $0 \leq i < k$
\begin{equation}
s(i)-2\Sigma_{j\geq i+1}s(j) \leq \rho(-i-1) \leq s(i)-1
\end{equation}
\end{proposition}
\begin{proof}
Since we are in the case $c_1=-1$, the restriction of $\mathcal{E}$ to the general plane is stable (\cite[Barth, Theorem 3]{Barth1977}). So the map $M_{-l-1} \to M_{-l}$ is injective for every $l \geq 0$. Define $N_{-l}$ by the exact sequence $0 \to M_{-l-1} \to M_{-l} \to N_{-l} \to 0$ and call $n_{-l}$ the dimension of $N_{-l}$. Let $x \in H^0(\mathcal{O}_{\mathbb{P}^2}(1))$, we have a natural map $N_{-l-1} \overset{x}{\to} N_{-l}$, consider then the map $N_{-l-1} \otimes H^0(\mathcal{O}_{\mathbb{P}^2}(1)) {\to} N_{-l}$. From the proof of \cite[Hartshorne, Theorem 5.3]{Hartshorne1980} we have that this map has image of dimension $ > n_{-l-1}$. Let $N=\bigoplus_{k\in \mathbb{Z}}N_k$, then $\rho(-l)$ is also the number of generators for $N$ in degree $-l$ which results for $l \geq 1$
\begin{equation}
\rho(-l)=n_{-l}-n_{-l-1}-1
\end{equation}
But $n_{-l}-n_{-l-1}=s(l-1)$, hence
\begin{equation}
\rho(-l) \leq s(l-1)-1
\end{equation}
Notice that $N_{-l-1} \otimes H^0(\mathcal{O}_{\mathbb{P}^2}(1)) {\to} N_{-l}$ has image $\leq 3n_{-l-1}$ since $h^0(\mathcal{O}_{\mathbb{P}^2}(1))=3$. Hence $N_{-l}$ must contain at least $n_{-l}-3n_{-l-1}$ minimal generators for $N$, equation (\ref{spectrum}) gives the inequality.
\end{proof}
We have seen that the a monad for $\mathcal{E}$ must be of the form
\begin{equation}
\bigoplus_{i=1}^{r} \mathcal{O}_{\mathbb{P}^3}(-c_i-1) \to \bigoplus_{i=1}^{r+1} (\mathcal{O}_{\mathbb{P}^3}(-b_i-1) \oplus \mathcal{O}_{\mathbb{P}^3}(b_i)) \to \bigoplus_{i=1}^{r} \mathcal{O}_{\mathbb{P}^3}(c_i)
\end{equation}
For the sake of simplicity denote it by
\begin{equation}
\mathcal{V}_0^*(-1) \to \mathcal{V}_1 \to \mathcal{V}_0
\end{equation}
Next proposition gives us a tool to limit the negative degrees appearing in the summands of $\mathcal{V}_0$:
\begin{proposition}
Suppose $\mathcal{V}_0$ has $r$ summands with degrees $\leq l$. Then $\mathcal{V}_1$ must contain at least $r+3$ summands with degrees $\geq -l$
\end{proposition}
\begin{proof}
$\mathcal{V}_0^*(-1)$ has $r$ summands with degrees $\geq -l-1$, and since the monad is minimal these must embed into the summand of $\mathcal{V}_1$ consisting of terms with degrees $\geq -l$. The quotient of this embedding has rank 3 or more
%\textcolor{red}{não achei a ref que foi passada pra essa conclusão}
\end{proof}
With equation (\ref{maxmod}) we can construct a monad for the first three examples:
\begin{example}(Disjoint union of conics)
Let $Y$ and $\mathcal{E}$ be the curve and the vector bundle of example (\ref{exemplobeilinson}). We already know that $\chi(\mathcal{E})=p_0^{r-1}$ so using the notation from equation (\ref{maxmod}) we have $k=0$ and $s(k)=r-1$. Then $\rho(-1)=r-1$ and hence we know that there is a $c_i=1$ which appears $r-1$ times in $\mathcal{V}_0$. Since $0$ was the largest integer in the spectrum, there can be no integer greater than $1$ in $\mathcal{V}_0$, so we can only complete it with $\mathcal{O}_{\mathbb{P}^3}$'s. But if we had any summand $\mathcal{O}_{\mathbb{P}^3}$ in $\mathcal{V}_0$ the monad wouldn't be minimal, hence we have an unique monad
\begin{equation}
 \mathcal{O}_{\mathbb{P}^3}(-2)^{r-1} \to \mathcal{O}_{\mathbb{P}^3}(-1)^{r} \oplus \mathcal{O}_{\mathbb{P}^3}^{r} \to \mathcal{O}_{\mathbb{P}^3}(1)^{r-1}
\end{equation}
\end{example}
\begin{example}
Let $\mathcal{E}$ be the vector bundle of example $\ref{exemplobeilinson2}$. We already have $\chi(\mathcal{E})=p_0p_1...p_{m-2}$ where $c_2=2m-2$ and $m \geq 2$. Then $k=m-2$ and $\rho(-m+1)=s(k)=1$, so $\mathcal{O}_{\mathbb{P}^3}(m-1)$ appears only once in $\mathcal{V}_0$. Using proposition (\ref{desigualdade}) we get that $\mathcal{V}_0=\mathcal{O}_{\mathbb{P}^3}(m-1)$.
\begin{equation}
\mathcal{O}_{\mathbb{P}^3}(-m) \to \mathcal{O}_{\mathbb{P}^3}(a)\oplus\mathcal{O}_{\mathbb{P}^3}(-a-1)\oplus\mathcal{O}_{\mathbb{P}^3}(b)\oplus\mathcal{O}_{\mathbb{P}^3}(-b-1) \to \mathcal{O}_{\mathbb{P}^3}(m-1)
\end{equation}
We can compute $a$ and $b$ using the Chern polynomial $c_t$:
If $\mathcal{V}_0 \to \mathcal{V}_1 \to \mathcal{V}_2$ is a monad, using the associated exact sequences
\begin{equation}
0 \to \mathcal{K} \to \mathcal{V}_1 \to \mathcal{V}_2 \to 0
\end{equation}
and
\begin{equation}
0 \to \mathcal{V}_0 \to \mathcal{K} \to \mathcal{E} \to 0
\end{equation}
and using the fact that $c_t$ is multiplicative, we get $c_t(\mathcal{E})=\frac{c_t(\mathcal{V}_1)}{c_t(\mathcal{V}_0)c_t(\mathcal{V}_2)}$, we can then compare each coefficient since $c_1(\mathcal{E})=-1$, $c_2(\mathcal{E})=2m-2$. In our case the denominator will be $1-t-m(m-1)t^2$, and its inverse is $1+t+(1+m(m-1))t^2+(1+2m(m-1))t^3$. Multiplying this last polynomial with the numerator gives the polynomial $1-t+(m^2-m-b-b^2-a-a^2)t^2$, using again that $c_2=2m-2$ we are left with a diophantine equation
\begin{equation}\label{diophantine}
(m-1)(m-2)=b+b^2+a+a^2
\end{equation}
We already know the case $m=2$, so let $m=3$ $(c_2=4)$
\begin{equation}
2=b+b^2+a+a^2
\end{equation}
has 5 solutions, all of them give the same monad
\begin{equation}
\mathcal{O}_{\mathbb{P}^3}(-3) \to \mathcal{O}_{\mathbb{P}^3}(-2)\oplus\mathcal{O}_{\mathbb{P}^3}(-1)\oplus\mathcal{O}_{\mathbb{P}^3}\oplus\mathcal{O}_{\mathbb{P}^3}(1) \to \mathcal{O}_{\mathbb{P}^3}(2)
\end{equation}
The case $m=4$ $(c_2=6)$ gives
\begin{equation}
\mathcal{O}_{\mathbb{P}^3}(-4) \to \mathcal{O}_{\mathbb{P}^3}(-3)\oplus\mathcal{O}_{\mathbb{P}^3}(-1)\oplus\mathcal{O}_{\mathbb{P}^3}\oplus\mathcal{O}_{\mathbb{P}^3}(2) \to \mathcal{O}_{\mathbb{P}^3}(3)
\end{equation}
which suggests that a general solution is given by
\begin{equation}
\mathcal{O}_{\mathbb{P}^3}(-m) \to \mathcal{O}_{\mathbb{P}^3}(-m+1)\oplus\mathcal{O}_{\mathbb{P}^3}(-1)\oplus\mathcal{O}_{\mathbb{P}^3}\oplus\mathcal{O}_{\mathbb{P}^3}(m-2) \to \mathcal{O}_{\mathbb{P}^3}(m-1)
\end{equation}
In order to verify it just take $a=-1$ and $b=-m+1$.
\end{example}
\begin{example}
Let $\mathcal{E}$ be the vector bundle from example (\ref{exemplobeilinson3}). $\chi(\mathcal{E})=p_0^3p_1^2$, $c_2(\mathcal{E})=10$ and $\rho(-2)=s(1)=2$. Using proposition (\ref{desigualdade}) we get $\rho(-1) \leq 2$, so we need to analyze three different cases.\\
Case 1: if $\rho(-1)=0$ the monad will be
\begin{equation}
\mathcal{O}_{\mathbb{P}^3}(-3)^{2} \to \mathcal{O}_{\mathbb{P}^3}(a)\oplus\mathcal{O}_{\mathbb{P}^3}(-a-1)\oplus\mathcal{O}_{\mathbb{P}^3}(b)\oplus\mathcal{O}_{\mathbb{P}^3}(-b-1)\oplus \mathcal{O}_{\mathbb{P}^3}(c) \oplus \mathcal{O}_{\mathbb{P}^3}(-c-1) \to \mathcal{O}_{\mathbb{P}^3}(2)^{2}
\end{equation}
The same computation gives the diophantine equation
\begin{equation}
a+a^2+b+b^2+c+c^2=2
\end{equation}
Which has 24 solutions, all of them giving the same monad
\begin{equation}
\mathcal{O}_{\mathbb{P}^3}(-3)^{2} \to \mathcal{O}_{\mathbb{P}^3}(-2)\oplus\mathcal{O}_{\mathbb{P}^3}(-1)^2\oplus\mathcal{O}_{\mathbb{P}^3}^2\oplus \mathcal{O}_{\mathbb{P}^3}(1) \to \mathcal{O}_{\mathbb{P}^3}(2)^{2}
\end{equation}
Case 2: if $\rho(-1)=1$ we have
\begin{equation}
\begin{split}
\mathcal{O}_{\mathbb{P}^3}(-3)^{2} \oplus \mathcal{O}_{\mathbb{P}^3}(-2) \to  \mathcal{O}_{\mathbb{P}^3}(a)\oplus\mathcal{O}_{\mathbb{P}^3}(-a-1)\oplus\mathcal{O}_{\mathbb{P}^3}(b)\oplus\mathcal{O}_{\mathbb{P}^3}(-b-1)\oplus \mathcal{O}_{\mathbb{P}^3}(c) \oplus \mathcal{O}_{\mathbb{P}^3}(-c-1) \\ \oplus \mathcal{O}_{\mathbb{P}^3}(d) \oplus \mathcal{O}_{\mathbb{P}^3}(-d-1) \to  \mathcal{O}_{\mathbb{P}^3}(1) \oplus \mathcal{O}_{\mathbb{P}^3}(2)^{2}
\end{split}
\end{equation}
which gives the diophantine equation
\begin{equation}
a+a^2+b+b^2+c+c^2+d+d^2=4
\end{equation}
it has 96 solutions giving the monad
\begin{equation}
\mathcal{O}_{\mathbb{P}^3}(-3)^{2} \oplus \mathcal{O}_{\mathbb{P}^3}(-2) \to  \mathcal{O}_{\mathbb{P}^3}(-2)^2\oplus\mathcal{O}_{\mathbb{P}^3}(-1)^2\oplus \mathcal{O}_{\mathbb{P}^3}^2\oplus \mathcal{O}_{\mathbb{P}^3}(1)^2\to  \mathcal{O}_{\mathbb{P}^3}(1) \oplus \mathcal{O}_{\mathbb{P}^3}(2)^{2}
\end{equation}
Case 3: if $\rho(-1)=2$ we have
\begin{equation}
\begin{split}
\mathcal{O}_{\mathbb{P}^3}(-3)^{2} \oplus \mathcal{O}_{\mathbb{P}^3}(-2)^2 \to  \mathcal{O}_{\mathbb{P}^3}(a)\oplus\mathcal{O}_{\mathbb{P}^3}(-a-1)\oplus\mathcal{O}_{\mathbb{P}^3}(b)\oplus\mathcal{O}_{\mathbb{P}^3}(-b-1)\oplus \mathcal{O}_{\mathbb{P}^3}(c) \oplus \mathcal{O}_{\mathbb{P}^3}(-c-1) \\ \oplus \mathcal{O}_{\mathbb{P}^3}(d) \oplus \mathcal{O}_{\mathbb{P}^3}(-d-1) \oplus \mathcal{O}_{\mathbb{P}^3}(e) \oplus \mathcal{O}_{\mathbb{P}^3}(-e-1) \to  \mathcal{O}_{\mathbb{P}^3}(1)^2 \oplus \mathcal{O}_{\mathbb{P}^3}(2)^{2}
\end{split}
\end{equation}
gives the diophantine equation
\begin{equation}
a+a^2+b+b^2+c+c^2+d+d^2+e+e^2=6
\end{equation}
there are two monads satisfying the equation:
\begin{equation}
\begin{split}
\mathcal{O}_{\mathbb{P}^3}(-3)^{2} \oplus \mathcal{O}_{\mathbb{P}^3}(-2)^2 \to  \mathcal{O}_{\mathbb{P}^3}(-3)\oplus\mathcal{O}_{\mathbb{P}^3}(-1)^4\oplus\mathcal{O}_{\mathbb{P}^3}^4\oplus\mathcal{O}_{\mathbb{P}^3}(2)\to  \mathcal{O}_{\mathbb{P}^3}(1)^2 \oplus \mathcal{O}_{\mathbb{P}^3}(2)^{2}
\end{split}
\end{equation}
which cannot occur, since we are assuming the monad to be minimal, and
\begin{equation}
\mathcal{O}_{\mathbb{P}^3}(-3)^{2} \oplus \mathcal{O}_{\mathbb{P}^3}(-2)^2 \to  \mathcal{O}_{\mathbb{P}^3}(-2)^3\oplus\mathcal{O}_{\mathbb{P}^3}(-1)^2\oplus\mathcal{O}_{\mathbb{P}^3}^2\oplus\mathcal{O}_{\mathbb{P}^3}(1)^3\to  \mathcal{O}_{\mathbb{P}^3}(1)^2 \oplus \mathcal{O}_{\mathbb{P}^3}(2)^{2}
\end{equation}
\end{example}
%\begin{example}
%	Let $Y$ be the projection of a canonical curve of genus $g$ in $\mathbb{P}^{g-1}$ into $\mathbb{P}^3$. Then $deg(Y)=2g-2$ and $\omega_{Y} \cong \mathcal{O}_Y(1)$ so as before $Y$ corresponds to a bundle $\mathcal{E}'$ with $c_1=5$ and $c_2=2g-2$. The normalized bundle $\mathcal{E}$ has Chern classes $c_1=-1$ and $c_2=2g-8$. We will have an exact sequence
%	\begin{equation}
%	0 \to \mathcal{O}_Y(-3) \to \mathcal{E} \to \mathcal{I}_Y(2) \to 0
%	\end{equation}
%	So $\mathcal{E}$ is stable if and only if $Y$ is not contained in a quadric surface, which is the case for $g \geq 5$ \textcolor{red}{como mostrar?}.
%\end{example}
% ---
% --------------------
% FIM DOS EXEMPLOS
% --------------------
% ----------------------------------------------------------
% CONSIDERAÇÕES FINAIS
% Finaliza a parte no bookmark do PDF para que se inicie o
% bookmark na raiz e adiciona espaço de parte no Sumário
%\phantompart
% ----------------------------------------------------------
% ---------------------------------------------------------------------------------
% %%%%%%%%%%%%%%%%%%%%%%%%%% FIM DOS ELEMENTOS TEXTUAIS %%%%%%%%%%%%%%%%%%%%%%%%%%
% ---------------------------------------------------------------------------------
% ---------------------------------------------------------------------------------
% %%%%%%%%%%%%%%%%%%%%%%% INÍCIO DOS ELEMENTOS PÓS-TEXTUAIS %%%%%%%%%%%%%%%%%%%%%%%
% ---------------------------------------------------------------------------------
% ----------------------------------------------------------
\postextual
% ----------------------------------------------------------
% ----------------------------------------------------------
% REFERÊNCIAS
\bibliographystyle{plain}
\bibliography{bibliografia}
% ----------------------------------------------------------
% ----------------------------------------------------------
% GLOSSÁRIO
% ----------------------------------------------------------
%\phantompart
%\cleardoublepage
%\phantomsection
%\addcontentsline{toc}{chapter}{\glossaryname}
%\printglossaries
% ----------------------------------------------------------
% ----------------------------------------------------------
% APÊNDICES
%\begin{apendicesenv}
% Imprime uma página indicando o início dos apêndices
%\partapendices
%\chapter{Quisque libero justo}

\lipsum[50]

\chapter{Nullam elementum urna vel imperdiet sodales elit ipsum pharetra ligula
ac pretium ante justo a nulla curabitur tristique arcu eu metus}

\lipsum[55-57]
%\end{apendicesenv}
% ----------------------------------------------------------
% ----------------------------------------------------------
% ANEXOS
%\begin{anexosenv}
% Imprime uma página indicando o início dos anexos
%\partanexos
%\chapter{Morbi ultrices rutrum lorem.}

\lipsum[30]

\chapter{Cras non urna sed feugiat cum sociis natoque penatibus et magnis dis
parturient montes nascetur ridiculus mus}

\lipsum[31]

\chapter{Fusce facilisis lacinia dui}

\lipsum[32]
%\end{anexosenv}
% ----------------------------------------------------------
%-----------------------------------------------------------
% INDICE REMISSIVO
%-----------------------------------------------------------
\phantompart
\printindex
%-----------------------------------------------------------
% ---------------------------------------------------------------------------------
% %%%%%%%%%%%%%%%%%%%%%%%%%% FIM DOS ELEMENTOS TEXTUAIS %%%%%%%%%%%%%%%%%%%%%%%%%%
% ----------------------------------------------------------------------------------
% -----------------------------------------------------------------------------------------------
% %%%%%%%%%%%%%%%%%%%%%%%%%%%%%%%%%%%%%% FIM DO DOCUMENTO %%%%%%%%%%%%%%%%%%%%%%%%%%%%%%%%%%%%%%
% -----------------------------------------------------------------------------------------------
\end{document}